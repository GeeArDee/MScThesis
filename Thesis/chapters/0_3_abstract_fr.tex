% LTeX: language=fr

\begin{plainchp}{Résumé}
    \addcontentsline{toc}{chapter}{Résumé}

    La propulsion laser-thermique (LTP) utilise des lasers pour chauffer le gaz propulseur, générant ainsi de la poussée avec une impulsion spécifique potentiellement plus grande que les moteurs fusées traditionnels. Dans cette thèse, deux propulseurs LTP à l'échelle du laboratoire ont été testés : Version 1 (V1) et Version 2 (V2). Le processus de conception du propulseur amélioré V2 est également présenté. Ces deux propulseurs seront alimentés par un laser à fibre de \qty{1.07}{μm} ayant une puissance ondes continues (CW) de \qty{300}{W} et une puissance à ondes quasi continues (QCW) de \qty{3}{kW}. L'argon est utilisé comme gaz propulseur à une pression de \qty{20}{bar}. L'amorçage par étincelle du plasma soutenu par laser (LSP) QCW a été implémenté avec succès dans V1 et V2. L'ensemencement en \ce{NO2} de l'argon à des pressions partielles comprises entre \qtyrange{.12}{.55}{bar} a montré que le gaz absorbait plus du double de l'énergie laser par rapport à de l'argon pur. Pour augmenter le flux laser vers le plasma, un système optique composé de deux lentilles a été conçu. Un LSP CW a été obtenu avec V2, d'une durée de \qty{85.1}{ms}. Cela représente une durée de vie 1.7 fois plus longue que la longueur d'impulsion QCW maximale de \qty{50.0}{ms} à cette puissance. La poussée moyenne à froid de V2 était de \qty{0.96}{N}. Afin d'interpréter les résultats expérimentaux, un modèle de transfert de chaleur zéro dimension (0D) a été écrit en Python, en utilisant le Bremsstrahlung comme mécanisme de rayonnement. Enfin, des pièces permettant d'améliorer le propulseur V2 sont présentées pour éventuellement permettre un essai CW à chaud du propulseur.
    
\end{plainchp}