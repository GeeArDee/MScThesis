\begin{plainchp}{Abstract}
    \addcontentsline{toc}{chapter}{Abstract}

    A laser-sustained plasma (LSP) generator capable of operating as a laser-thermal thruster was designed and tested, powered by a \num{3}-\unit{kW} \num{1070}-\unit{nm} fiber-optic laser, and operated with argon gas at pressures of \qtyrange{3}{20}{bar}. This thesis documents its design process and early test results. The laser absorption, radiated spectrum, and change in static pressure was recorded, and high-speed video footage of the LSP was acquired. Of special interest is the heat-deposition efficiency into the working gas resulting from LSP ignition, which was estimated by tracking the pressure change in the test section. Laser transmission through the absorbing plasma was also measured, to estimate the magnitude of two energy loss mechanisms: incomplete laser absorption and radiation of heat to the walls of the test section. Characterizing and minimizing these losses would be critical in the realization of a practical and efficient laser-thermal propulsion (LTP) thruster, as most of the laser energy should be deposited as heat in the propellant to maximize thrust chamber temperature and therefore thruster specific impulse.

    This report first introduces the concept of laser-thermal propulsion, highlighting its potential as a high-specific impulse, high-thrust deep-space pro\-pulsion system competing with proposed nuclear-thermal thruster concepts. A brief summary of past literature on LTP is provided: first imagined in the 1970s, and studied intensively over the following two decades along with the physical mechanism powering the concept---laser-sustained plasma. Although LSP has been tested experimentally before, most studies used \ce{CO_2} lasers operating at \qty{10.6}{\um}, while current thinking on directed-energy pro\-pulsion favors \num{1.06}-\unit{\um} fiber lasers. Furthermore, previous LTP prototypes have demonstrated the concept without attempting to optimize thruster performance to a level that could be deemed competitive. Updating experimental research with fiber-optic lasers and resuming LTP studies with the aim to maximize thruster performance is thus identified as a research gap. This study aimed to design a facility to study LSP and LTP in the lab, and to establish a baseline heat-deposition efficiency measure against which further improvements and facility iterations can be compared.

    The design process of the test facility is first documented in detail, discussing available laser equipment and diagnostics apparatus. Top-level requirements for the test section and thrust stand are given. The apparatus built for an unrelated experiment and left unused is identified as a suitable candidate for retrofitting as an LSP generator, speeding up the realization of an experimental setup. Retrofitting, integration, test, and calibration activities are all described in detail. A discussion of LSP absorption modelling is given to provide a deeper understanding of the relevant parameters affecting absorption, namely laser wavelength and gas pressure. \replaced{Simple models for LSP sizing, heat deposition in the test section, and expected thrust performance are also presented.}{Knowing the variation in LSP absorption coefficient is also useful to estimate the peak temperature of the LSP, as both metrics are correlated.}

    Then follows a summary of the results obtained from early experiments with the facility, starting with exploratory ignition experiments. The spark-ignition system considered for use in the project was more difficult to use than expected, as successful and reliable ignition was found to require a consistent arc path, which was not the case with the final spark-plug design. LSP was nevertheless ignited successfully using this method a few times, allowing some measure of its laser-absorption, found to be 79\% on average at \qty{20}{bar}. Ignition by thermionic emission from a tungsten wire was found to provide much more consistent ignition, though this prevented the measurement of transmitted laser energy. Power threshold experiments commonly performed in LSP literature were reproduced, showing that this LSP facility matched or outperformed results of past studies, with LSP sustained at \qty{5}{bar} with as low as \qty{600}{W} of laser power, 40\% less than would be estimated from past literature. Heat deposition into the working gas was estimated by tracking the change in static pressure inside the test section, then using the ideal gas law to relate it to the added heat. Heat-deposition efficiency appears to be consistent across pressures at approximately 15\%. Finally, spectroscopic temperature measurement was attempted using the Boltzmann plot method, but the analyses have not yielded realistic temperature estimates, likely due to methodology issues.

    Some flowing/thrust experiments were also attempted. Unfortunately, the retrofitted apparatus was not optimized for thruster operation, and its excessive weight was a major obstacle to thrust measurement. The thrust stand designed by a collaborating team of students was not able provide consistent thrust readings. Pressure and spectral data was acquired for LSP in bulk flow velocities ranging from \qtyrange{0.88}{1.8}{m.s^{-1}}, but exhibited little to no difference from the static case. The growth and size of the LSP was observed to change under flowing conditions: as seen in the literature, the LSP front speed was slower, and the tail extended further downstream, resulting in an LSP located closer to the laser focus.

    Although the resulting laboratory model is not an optimized LTP thruster prototype, the facility provides a platform upon which future, more targeted, and more rigorous experiments can be performed, following up on the results presented in this report. The lessons learned in designing the facility and early tests will be invaluable in the design of the next iteration of laser-thermal thruster laboratory models at McGill University.

\end{plainchp}