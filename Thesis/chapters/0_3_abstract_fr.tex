% LTeX: language=fr

\begin{plainchp}{Résumé}
    \addcontentsline{toc}{chapter}{Résumé}

    Les progrès récents de la commercialisation de l'espace ont rendu l'accès à l'orbite terrestre basse (LEO) plus abordable. Cependant, pour des missions ambitieuses telles que le transit humain rapide vers Mars, les méthodes de propulsion existantes ne sont pas à la hauteur. La propulsion laser-thermique (LTP) apparaît comme une solution prometteuse, utilisant des lasers pour chauffer le gaz propulseur et générer une poussée avec une efficacité supérieure à celle des moteurs-fusées traditionnels. L'objectif de cette recherche est de mener une série d'expériences avec un propulseur LTP de laboratoire alimenté par un laser à fibre de 1070 nm. Cette recherche vise à mesurer les paramètres de performance critiques tels que la poussée et l'impulsion spécifique, et à caractériser le cœur de chauffage du plasma entretenu par le laser. Ce projet s'appuie sur les connaissances théoriques tirées d'études antérieures et cherche à repousser les limites de la recherche expérimentale sur ce système de propulsion. Le laser générera un plasma qui sera analysé en profondeur à l'aide de divers instruments de diagnostic. La phase initiale de la recherche a démontré avec succès la fiabilité de l'allumage du plasma d'argon, préparant ainsi le terrain pour d'autres expériences. Celle-ci a commencé par le fonctionnement pulsé d'un propulseur de première génération. Les phases suivantes ont consisté à concevoir un propulseur de deuxième génération et à le tester en fonctionnement continu.
    
\end{plainchp}