\chapter{Conclusion}

    \section{Summary}

        % Say how I answered the research objectives here
        % Conclusion: did spark initiation V1, validated, characterised V2, spark V2, and CW in V2.

        Static LSP validation showed V1 spark initiation of QCW LSP, V2 spark initiation of QCW LSP, and an XXX ms CW LSP.

        Cold flow thruster characterization showed thrusts of about \qty{1}{N} at \qty{20}{bar}, and the need for a new thrust stand design.

        Flowing QCW LSP showed:

        This thesis lays the groundwork for future thrust tests with longer plasma lifetimes and better thrust measurement, which would show the thrust added by the laser.

    \section{Future Work}

        % make complete sentences and em dashes for itemization, NOT bullet points
        Near term: pressure test the 5 new parts (enumerate) built and start CW shots again.

        \begin{itemize}
            \item Measure mass flow rate $\dot m$ of thruster with bubble flow meter
            \item Design a new thrust stand for more repeatable measurements (ex: a rotating arm thrust stand)
            \item Model the LSP in CFD [?]
            \item Implement spectroscopic measurements to determine temperature of plasma in our setup
        \end{itemize}

        Long term, to ultimately test is LTP is a viable means of space propulsion:

        \begin{itemize}
            \item Use a more powerful laser. Instead of 300 W CW, use 3kW CW.
            \item Move to hydrogen as a propellant
            \item Complete thrust tests in vacuum
        \end{itemize}

        %(be sure to talk about these in text before, where these would be improvements and why)

        
        
        
        
        
        
        
        
 
