\chapter{Introduction} \label{chp:intro}
    % AIAA Talk written out
    In 2016, the Breakthrough Starshot initiative was proposed based on the work of Professor Philip Lubin at UC Santa Barbara. This mission involves beaming \qty{1}{g} space probes to \qty{20}{\%} the speed of light, using ground-based laser arrays.

    This could be enabled by a Moore's law in fiber laser technology, with a rapid doubling of power and a similar exponential decrease in costs. The era of giant phased-array lasers is coming! As a near-term stepping stone using a smaller array, we can couple the laser to a gas, reducing efficiency but increasing thrust. This would allow rapid interplanetary transfers and is called Laser-Thermal Propulsion (LTP).

    The system architecture is similar to Breakthrough Starshot: a \qty{10}{m} laser array beams \qty{100}{MW} of power to an orbiting spacecraft for injection burns. Our team proposed a conceptual design of an LTP spacecraft in 2022 in Acta Astronautica [CITE]. With a \qty{1000}{kg} payload, about \qty{6}{kN} of thrust and \qty{3000}{s} of $I_{sp}$, we do a \qty{1}{h} maneuver to get \qty{14}{km/s} of delta-V and reach Mars in 45 days.

    In the thrust chamber of the vehicle, hydrogen is introduced. The laser is focused inside, creating a plasma core. Colder hydrogen flows around it and is heated by the plasma to \qty{10000}{K}. It is then exhausted though a conventional converging-diverging nozzle, imparting thrust. My research concentrates on the laser-plasma coupling, and the heating of the bulk gas.