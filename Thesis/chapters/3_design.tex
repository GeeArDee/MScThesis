\chapter{Facility design}
    \section{Version 1} \label{sec:design_v1}

        The first generation thruster, called Version 1 (V1), was designed by Emmanuel Duplay during his master's project. This process can be found in his thesis \cite{duplayArgonLaserPlasmaThruster2024a}. It proved to be a dependable prototype, repurposed from a previous unrelated experiment. However, it presented problems that required a second generation prototype to be designed and manufactured.

        \subsection{Capabilities}

            298 recorded pulsed laser shots were conducted with V1, exploring the power-pressure threshold, wire ignition and spark ignition.

        \subsection{Issues}

            Too heavy -> too much friction on rail for thrust tests. This was mitigated in part by a rope system.

            Rubber seals were exposed to laser path during continuous lasing with a lower focal length lens (picture), severely burning them in the only CW test conducted. A shorter test section designed for a \qty{100}{mm}focal length lens would solve this.







    \section{Version 2} \label{sec:design_v2}

        To improve upon the V1 facility, an entire redesign was done over the course of two semesters with a MECH 463 Capstone team. The students in this team (Alain Halbouny, Jade Tabbara, Karl Atallah, Serge Rubeiz and Shady Tawil) successfully delivered a Version 2 (V2) purpose-built LSP thruster at the end of April 2024.

        \subsection{Requirements}

            The following requirements were developed for the capstone team to lead them in the design of the V2 thruster.

            % Put this in a nice big font
            Objective: Detect a measurable difference in thrust between an argon cold gas thruster and an argon “hot gas” thruster, heated by a laser supported plasma (LSP).
            %%%

            \begin{enumerate}
                \item Laser thruster
                \begin{enumerate}
                    \item A \qty{300}{W} Continuous Wave (CW) \qty{1070}{nm} laser shall sustain the plasma (Nominal power \qty{300}{W}, actual max power \qty{350}{W})
                    \item The thruster shall have a minimum safe “hot” operation time of \qty{30}{s}
                    \begin{enumerate}
                        \item In the event of failed LSP ignition, the thruster shall safely absorb the total laser power for at least \qty{10}{s}
                    \end{enumerate}
                    \item An optical path shall be present to let the laser into the thruster, utilizing a \qty{100}{mm} focal length lens at minimum and a collimated beam with a maximum diameter of \qty{30}{mm}
                    \begin{enumerate}
                        \item The optical components shall not be damaged by the laser flux
                    \end{enumerate}
                    \item Argon shall be used as the working fluid
                    \begin{enumerate}
                        \item The argon feed gas shall be at room temperature
                    \end{enumerate}
                    \item A gas feed path shall bring argon gas into the thruster
                    \begin{enumerate}
                        \item The gas feed shall be choked at the thruster inlet
                        \item The gas feed shall be evenly distributed in the thruster
                    \end{enumerate}
                    \item The mass flow rate of the argon gas shall be measured and controlled by interchangeable upstream choked orifices
                    \item The maximum allowable operating pressure (MAOP) of the thruster shall be 50 bar
                    \begin{enumerate}
                        \item The nominal pressure of the thruster shall be 25 bar
                    \end{enumerate}
                    \item A converging-diverging exhaust nozzle shall be designed to accelerate the gas to a supersonic speed
                    \begin{enumerate}
                        \item The nozzle shall be easily changeable
                    \end{enumerate}
                    \item A 1/8" NPT port for a pressure transducer shall be present along the thruster
                    \item An optical port shall be present for spectrometry measurements of the plasma
                    \item The thruster shall be installed on a thrust stand (See section 3. Thrust stand)
                \end{enumerate}
                \item Ignition system/electrical
                \begin{enumerate}
                    \item The LSP shall be ignited by an electrical spark
                    \item The spark gap shall be measurable, controllable and repeatable
                    \item The spark shall be generated by an AEM 30-2853 High Output Smart Coil, supplying \qty{41}{kV} with up to \qty{118}{mJ}
                    \item All parts of the thruster and thrust stand shall be directly or indirectly connected to a common electrical ground
                \end{enumerate}
                \item Thrust stand
                \begin{enumerate}
                    \item The thrust stand shall measure thrust on the order of \qtyrange{0.1}{5}{N}
                    \item The thrust stand shall minimize friction losses
                    \item The thrust stand shall be securely fixed using standard optical breadboard mounting hardware
                \end{enumerate}
            \end{enumerate}


        \subsection{Sizing the double choked LTP thruster}

            \begin{figure}[h]
            \centering
            \includegraphics[width=0.8\linewidth]{assets/3 design/double choked sizing.png}
            \caption{\label{fig:frog}Schematic of an LTP thruster.}
            \end{figure}
            
            Let's pose that we have a 300 W power input (laser) and want to run the LTP experiment at 25 bar, with a 50 bar feed pressure. We want the hot gas operation (laser on) to increase the gas' exit velocity to twice that of the cold gas operation (laser off). We will determine together the gas mass flow rate and the diameter of the two orifices needed to choke the flow.
            
            \subsubsection{Cold Ar thruster}
            $C_0$ choked, without nozzle is speed of sound: $323 \:m/s$ This is at ambient temperature (300 K), as we have no laser energy to heat the gas in this case. With a nozzle, we accelerate the gas by $\approx 2$ times. The $v_{exit}$, which is our main performance parameter, is therefore: 
            \[v_{exit} = 646\: m/s\]
            
            \subsubsection{Hot Ar thuster}
            Taking the previous $v_{exit}$ and ionizing the whole flow, we hope that our efficiency is doubled. This gives a $v_{exit}\approx 1300\:m/s$. What nozzle throat size is necessary for this $\dot m$ with $p_{chamber} = 25\: bar$? We know that $MW_{Ar} = 40 \: g/mol$. From Fluids 2, the speed of sound is $c = \sqrt{\gamma R T}$. As we want to double the speed of sound, we are multiplying the temperature by 4! From Thermo 1:
            \[Power = \dot m (h_2-h_1)
            = \dot m C_p (T_2-T_1)\]
            The $C_p$ of argon is $0.520\:kJ/(kg-K)$ Find the $\dot m$ that we can expect in g/s.\vspace{60mm}
            
            Again from Fluids 2, we have the mass flow rate (of an isentropic flow) described by Fliegner's formula:
            \[\frac{\dot m}{A} = p_0\sqrt{\frac{\gamma}{T_0 R}}\frac{M}{(1+\frac{\gamma-1}{2}M^2)^{(\frac{\gamma+1}{2(\gamma-1)})}}\]
            We have a $\gamma = \frac{C_p}{C_v} = 1.666$ for Argon and choked flow at the nozzle. Find the area and the diameter of the circular nozzle.\vspace{170 mm}
            
            We can repeat these calculations for the feed orifice, with the same $\dot m$, a pressure of 50 bar and ambient temperature. This gives us an orifice size of about 0.2 mm, or 10 thou in freedom units.

        \subsection{Capstone team deliverable}
            
            \subsubsection{Installation}

                Once the 

        \subsection{Improvements on Capstone design}

            \subsubsection{Electrodes}
                The \qty{44}{kV} wire having burst during pressure tests, 


    
    \section{title}
