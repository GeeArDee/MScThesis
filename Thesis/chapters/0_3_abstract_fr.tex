% LTeX: language=fr

\begin{plainchp}{Résumé}
    \addcontentsline{toc}{chapter}{Résumé}

    \todo{REDO}
    Les progrès récents de la commercialisation de l'espace ont rendu l'accès à l'orbite terrestre basse (LEO) plus abordable. Cependant, pour des missions ambitieuses telles que le transit humain rapide vers Mars, les méthodes de propulsion existantes ne sont pas à la hauteur. La propulsion laser-thermique (LTP) apparaît comme une solution prometteuse, utilisant des lasers pour chauffer le gaz propulseur et générer une poussée avec une plus grande efficacité que les moteurs fusées traditionnels. L'objectif de cette thèse était de tester un propulseur LTP Version 1 (V1) à l'échelle du laboratoire pour les vols spatiaux interplanétaires, ainsi que de concevoir et de tester un propulseur Version 2 (V2) amélioré. Ces deux propulseurs seront alimentés par un laser à fibre de \qty{1.07}{μm} ayant une puissance ondes continues (CW) de \qty{300}{W} et une puissance à ondes quasi continues (QCW) de \qty{3}{kW}. L'argon est utilisé comme gaz propulseur à une pression de \qty{20}{bar}. L'amorçage par étincelle du plasma soutenu par laser (LSP) QCW a été implémenté avec succès dans V1 et V2. L'ensemencement de l'argon en \ce{NO2} à des pressions partielles comprises entre \qtyrange{.12}{.55}{bar} a montré que le gaz absorbait plus du double de l'énergie laser par rapport à de l'argon pur. Un LSP CW a été obtenu avec V2, d'une durée de \qty{85.1}{ms}. Cela représente une durée de vie 1.7 fois plus longue que la longueur d'impulsion QCW maximale de \qty{50.0}{ms} à cette puissance. La poussée moyenne à froid de V2 était de \qty{0.96}{N}. Des pièces permettant d'améliorer encore V2 sont présentées afin de permettre éventuellement un essai CW à chaud du propulseur.
    
\end{plainchp}