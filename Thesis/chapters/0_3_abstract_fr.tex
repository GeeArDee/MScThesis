% LTeX: language=fr

\begin{plainchp}{Résumé}
    \addcontentsline{toc}{chapter}{Résumé}

    Pour des missions spatiales ambitieuses comme le transit rapide d'humains vers Mars, les méthodes de propulsion conventionnelles ne sont pas à la hauteur. La propulsion laser-thermique (LTP) utilise des lasers pour chauffer le gaz propulseur, générant ainsi de la poussée avec une impulsion spécifique potentiellement plus grande que les moteurs fusées traditionnels. Deux propulseurs LTP à l'échelle du laboratoire ont été testés dans cette thèse, désignés Version 1 (V1) et Version 2 (V2). Version 1 a permis les essais initiaux et la visualisation de la propagation de l'onde du plasma soutenu par laser (LSP). Un prototype d'un propulseur réel, Version 2 a été optimisé pour les essais de poussée. Le processus de conception du propulseur amélioré V2 est également présenté. Ces propulseurs ont été alimentés par un laser à fibre de \qty{1.07}{μm} ayant une puissance ondes continues (CW) de \qty{300}{W} et une puissance à ondes quasi continues (QCW) de \qty{3}{kW}. L'argon a été utilisé comme gaz propulseur à une pression de \qty{20}{bar}, sélectionné pour sa facilité d'ionisation. Utilisant une bobine de type automobile, l'amorçage par étincelle du LSP QCW a été implémenté avec succès dans V1 et V2. L'ensemencement de l'argon avec du dioxyde d'azote (\ce{NO2}) à des pressions partielles comprises entre 0.12~bar et 0.55~bar a montré que le gaz absorbait plus du double de l'énergie laser par rapport au propergol d'argon pur. Pour augmenter le flux laser vers le plasma, un système optique composé de deux lentilles a été conçu. Différentes lentilles ont été comparées à l'aide d'un logiciel de traçage de rayons (WinLens3D). Un LSP CW a été obtenu avec V2, d'une durée de \qty{85.1}{ms}. Cela représente une durée de vie 1.7 fois plus longue que la longueur d'impulsion QCW maximale de \qty{50.0}{ms} à cette puissance. La poussée moyenne à froid de V2 était approximativement \qty{1}{N}. Afin d'interpréter les résultats expérimentaux, un modèle de transfert de chaleur zéro dimension (0D) a été écrit en Python, en utilisant le Bremsstrahlung comme mécanisme de radiation. Enfin, des pistes pour améliorer encore le propulseur V2 et le banc d'essai de poussée sont présentées pour permettre à terme des mesures de poussée précises avec un fonctionnement CW.
    
\end{plainchp}