\section*{LSP Shot Procedure}\label{lsp-shot-procedure}

This checklist is used for starting up, using, then shutting down the
Laser-Sustained Plasma (LSP) experiment to perform an LSP test. This
checklist assumes the laser has already been aligned.

Not all components are used for all experiments (e.g., spark igniter).
Avoid setting them up if not necessary.

\subsubsection{STARTUP PROCEDURE}\label{startup-procedure}

\begin{enumerate}
\def\labelenumi{\arabic{enumi}.}

\item
  Power on triggering and monitoring station

  \begin{enumerate}
  \def\labelenumii{\arabic{enumii}.}
  
  \item
    Turn on delay generators, oscilloscope
  \item
    Oscilloscope trigger should be set to ``Normal'' mode
  \item
    Turn on laser in LOCAL mode
  \item
    Verify that timing delays have the correct value \textbf{and} unit
    (ms)
  \item
    Verify that delay generators, camera, and laser are connected
    according to wiring diagram
  \end{enumerate}
\item
  Setup camera

  \begin{enumerate}
  \def\labelenumii{\arabic{enumii}.}
  \item
    Turn on camera
  \item
    Connect to camera using Yellow Cat 6 cable
  \item
    Start PFV4 software and acquire live camera feed
  \item
    Calibrate sensor
  \item
    Remove lens cap
  \item
    Attach ND Filter and UV-IR Cut Filter to lens
  \item
    Adjust camera position and focus to frame the LSP ignition point
    (spark plug tips) at a 300 mm focal length

    {\color{cyan} Use low-light mode, set the lens aperture to f/4, set the ND
    Filter to minimum, and use additional lighting if needed}
  \item
    Turn off low-light mode, set the camera frame rate to 10000 fps
  \item
    Set the aperture to f/22 and the ND filter to the maximum level
  \end{enumerate}
\item
  Setup power meter

  \begin{enumerate}
  \def\labelenumii{\arabic{enumii}.}
  \item
    The black rubber protective cap should be on
  \item
    Plug in the power meter's fan power supply and ensure the fan is
    running
  \item
    Turn on the BLU emitter (blue button)
  \item
    Connect to the power meter on the PC Gentec-EO software using the
    Bluetooth dongle (attached to blue USB extension cord)

    {\color{cyan} For best results, leave the Bluetooth dongle within the test area}
  \item
    Remove the black rubber protective cap and allow the power meter's
    signal to stabilize
  \item
    Set the wavelength to 1070 nm and perform the zeroing procedure
  \item
    Set the power meter to ``SSE (J)'' (Single Shot Energy) mode
  \item
    Suggested: set the display mode to ``Statistics''
  \item
    If applicable, set your acquisition settings (filename, etc)
  \end{enumerate}
\item
  Setup spectrometer

  \begin{enumerate}
  \def\labelenumii{\arabic{enumii}.}
  
  \item
    Use a laser pointer and a 100 micron fiber attached to the fiber
    mount to ensure the fiber tip is pointed at the ignition point, then
    re-attach the spectrometer fiber (10 micron) to the fiber mount
  \item
    Connect to the spectrometer via USB and start the OceanView Lite
    software, in ``Quick View'' mode
  \item
    Click ``Create dark background spectrum''
  \item
    Set the integration time to 4 ms and the trigger mode to ``Edge''
  \end{enumerate}
\item
  Setup pressure transducer

  \begin{enumerate}
  \def\labelenumii{\arabic{enumii}.}
  
  \item
    If the transducer is already mounted, all that is needed is to turn
    on the signal conditioner and check that the Channel 1 indicator is
    green
  \end{enumerate}
\item
  Prepare test area

  \begin{enumerate}
  \def\labelenumii{\arabic{enumii}.}
  \item
    Ensure the laser protection panels (2) are installed over the beam
    path between the collimator and the test section
  \item
    Ensure that the collimator cap is \textbf{OFF}
    and no obstacle is present in the beam path. Use the guide laser to
    check.
  \item
    Pressurize the test section to the target pressure. If performing a
    flowing test, pressurize the feed lines upstream of the solenoid
    valve, and set the valve's safety switch to the FIRE mode

    {\color{cyan} Before pressurizing to target pressure, evacuate the air in the
    test section by filling it with Argon to 5 bar then venting it to
    1.5 bar, repeating this process three times.}

    \colorbox{Goldenrod}{\textbf{!}} \textbf{The
    laser windows are rated for a maximum internal pressure of 20 bar.
    Do not exceed this
    pressure.}
    Some tolerance for overpressure (\~1~bar) is available
    in order to let the system stabilize to 20 bar, but do not run tests
    in overpressure conditions. \emph{Destructive testing has not been
    performed to determine the actual failure pressure.}
  \item
    Plug in the spark igniter to the mains

    \colorbox{Goldenrod}{\textbf{!}} The igniter is now \textbf{ON} and will spark when receiving a
    signal
  \item
    Exit the area enclosed by the laser safety curtains and close them
  \end{enumerate}
\end{enumerate}

The experiment is now ready to be run.

\subsubsection{RUNNING THE EXPERIMENT}\label{running-the-experiment}

\begin{enumerate}
\def\labelenumi{\arabic{enumi}.}
\item
  Perform a final check on the control station to verify the timings and
  connections of the delay generators, oscilloscope, and laser.
\item
  Prepare the laser

  \begin{enumerate}
  \def\labelenumii{\arabic{enumii}.}
  \item
    Restart the laser in REMOTE mode
  \item
    Connect to the laser via the router, using the Black Cat 6 cable
  \item
    Use the laser's web interface to set up the pulse. Check the
    following settings:

    \texttt{HW\ Emission\ Control} should be ENABLED

    \texttt{Pulse\ Mode} should be ENABLED

    {\color{cyan}For more information on the web interface, consult the laser user
    guide}
  \item
    Set the pulse power setpoint to the desired value
  \item
    Set the pulse duration to the desired value
  \end{enumerate}
\item
  Connect to camera using Yellow Cat 6 cable, and confirm connection in
  PFV4
\item
  Click ``Record'' in PFV4. The button should read ``Ready''
\item
  Set the spectrometer save settings by clicking ``Configure graph
  saving'' in OceanView, entering the appropriate LSP shot identifier
  code as the BaseName, click ``Apply''
\item
  Click the ``Save graph to files'' icon in OceanView---this should turn
  the button red.
\item
  \textbf{All
  personnel present in the
  laboratory,}
  regardless of their involvement in the experiment,
  must
  equip laser safety goggles rated for 1070 nm beyond this
  step
\item
  Turn on the laboratory's laser warning light (confirm visually) and
  ensure the door is closed

  {\color{McGillRed}Entering/exiting the laboratory is not permitted beyond this step}
\item
  Disengage the laser's front-panel E-stop. Call out ``Safety OFF''.
\item
  Find the power supply switch wired in the back of the laser. Flick the
  switch ON then OFF. Call out ``Laser is ARMED''.

  {\color{cyan} This starts the laser's main power supply, this is indicated by a
  louder fan volume and the green button on the front panel being lit up}

  \colorbox{Goldenrod}{\textbf{!}} The laser is now \textbf{armed} - it will emit a laser pulse when
  the trigger signal is active
\item
  The experiment is ready to run, go through the following checklist
  before firing:

  \begin{itemize}
  
  \item[$\square$]
    Curtains are \textbf{CLOSED}
  \item[$\square$]
    Laboratory warning light is \textbf{ON}
  \item[$\square$]
    Laser is \textbf{ARMED}
  \item[$\square$]
    Camera is \textbf{READY}
  \item[$\square$]
    Power meter monitor is active and awaiting pulses
  \item[$\square$]
    All delay generators are \textbf{ON}
  \item[$\square$]
    Oscilloscope is \textbf{ON}
  \item[$\square$]
    \textbf{ALL LAB PERSONNEL IS WEARING LASER
    SAFETY GOGGLES}
  \end{itemize}
\item
  If performing a flowing test, use the valve switch near the control
  station to initiate flow. Allow for 5 seconds for the flow to
  stabilize, or up to 45 seconds for the pressure transducer signal to
  return to 0.
\item
  You may press the \texttt{MAN\ TRIG} button to emit a laser pulse.
  Watch the ceiling above the test area to spot the flash of a
  successful LSP ignition
\item
  Press the front panel E-stop to safe the laser. Call out ``SAFE''.
\item
  Regardless of ignition, the camera will have recorded footage. To
  perform a new shot, resume from step 4
\end{enumerate}

{\color{cyan} If at any point after step 6, someone must remove their safety
glasses, enter, or leave the lab, press the E-stop to safe the laser.
Resume procedure from step 5.}

\subsubsection{SHUTDOWN PROCEDURE}\label{shutdown-procedure}

\begin{enumerate}
\def\labelenumi{\arabic{enumi}.}
\item
  Press the laser's front-panel E-stop

  {\color{cyan} Laboratory personnel is now free to remove their laser safety
  glasses, and can freely enter/leave the lab}
\item
  Disable laser warning light
\item
  Open the laser safety curtains
\item
  Unplug the igniter
\item
  Vent the test section
\item
  Shut off the camera
\item
  Shut off the power meter, unplug its fan, and place the rubber
  protective cap back on
\item
  Screw on the collimator cap
\item
  Switch off the delay generators and the oscilloscope
\item
  Switch off the laser, place the keys in the ``Miscellaneous'' drawer
  of the component cabinet
\end{enumerate}