\chapter{Conclusion}

    \textcolor{red}{Conclusion is still a work in progress.}

    This thesis lays the groundwork for future thrust tests with longer plasma lifetimes and better thrust measurement, which would show the thrust added by the laser.

    Static LSP tests showed successful spark initiation of QCW LSP in V1 and V2, and an \qty{85.1}{ms} CW LSP. A new window mount for V2 was designed and used in static tests. \ce{NO2} seeding was also investigated in V1 as a way to increase energy retention by the gas, and resulted in a twofold increase in argon dynamic pressure. 

    Cold flow thruster characterization measured thrust of about \qty{1}{N} at \qty{20}{bar}. To get repeatable thrust measurements without significant hysteresis, the need for a new thrust stand design with less friction was made clear. Flowing QCW LSP tests led to a new nozzle design to prevent significant ablation due to laser energy that was not absorbed by the LSP. 
    
    Measurements of the mass flow rate $\dot m$ of V2 with the bubble flow meter are still to be completed. Once the new parts for V2 are installed and pressure tested, CW LSP tests can restart to explore longer plasma lifetimes.

    Long term, to ultimately test if LTP is a viable means of space propulsion:
    \begin{itemize}
        \item Use a more powerful laser
        \item Move to hydrogen as a propellant
        \item Complete thrust tests in vacuum
    \end{itemize}