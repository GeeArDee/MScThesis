\newcommand{\calculation}[3]{
    \begin{multline*}
        #1 = #2 \\ #1 = #3
    \end{multline*}
}

\chapter{Calculation examples} \label{chp:app_calc_ex}
    \section{Absorption coefficient} \label{sec:app_calc_ex_alpha}
        Calculation of the absorption coefficient $\alpha$ of an argon plasma for 1070-nm radiation at a temperature $T$ of \qty{15000}{K} and a pressure $p$ of \qty{10}{bar}.

        \subsection*{Givens}
            \begin{tabular}{r@{ = }l}
                $E_\text{ion}$      & \qty{15.76}{eV} \\
                $p$                 & \qty{10e+5}{Pa} \\
                $T$                 & \qty{15000}{K}  \\
                $\lambda$           & \qty{1.07e-6}{m}
            \end{tabular}
            \newcommand{\evalp}{\qty{10e+5}{Pa}}
            \newcommand{\evalT}{\qty{15000}{K}}

        \subsection{Electron density calculation}
            We begin with the calculation of electron density $n_\mathrm{e}$, using the Saha ionization equation:
            \begin{equation*}
                \frac{n_\mathrm{e}^2}{n_0-n_\mathrm{e}} = \frac{n_\mathrm{e}^2}{n_\mathrm{Ar}} = \frac{2}{\Lambda_\mathrm{th}^3}\frac{\mathcal{Z}_{\mathrm{Ar}^+}}{\mathcal{Z}_\mathrm{Ar}}\exp{\left(-\frac{E_\text{ion, Ar}}{k_\mathrm{B}T}\right)}
                \tag{\ref{eq:saha} revisited}
            \end{equation*}
            Calculating necessary parameters: the thermal DeBroglie wavelength $\Lambda_\mathrm{th}$, the initial atomic number density $n_0$, the partition function ratio $\mathcal{Z}_{\mathrm{Ar}^+}/\mathcal{Z}_\mathrm{Ar}$ (from NIST), and the ionization energy $E_\text{ion}$ in J.
            \begin{multline*}
                \Lambda_\mathrm{th} = \sqrt{\frac{2\pi\hbar^2}{m_\mathrm{e}k_\mathrm{B}T}} = \sqrt{\frac{2\pi(\evalhbar)^2}{(\evalme)(\evalkB)(\evalT)}} \\
                \Lambda_\mathrm{th} = \qty{6.0860e-10}{m}
            \end{multline*}
            \begin{multline*}
                n_0 = \frac{N_\mathrm{A}p}{R_\mathrm{u}T} = \frac{(\evalNA)(\evalp)}{(\evalRu)(\evalT)} \\
                n_0 = \qty{4.8286e24}{m^{-3}}
            \end{multline*}
            \begin{multline*}
                \frac{\mathcal{Z}_{\mathrm{Ar}^+}}{\mathcal{Z}_\mathrm{Ar}} = \frac{(5.74)}{(1.02)} \\ 
                \frac{\mathcal{Z}_{\mathrm{Ar}^+}}{\mathcal{Z}_\mathrm{Ar}} = 5.627
            \end{multline*}
            \calculation{E_\mathrm{ion}}{
                (\qty{15.76}{eV})\left(\frac{\qty{1.60218e-19}{J}}{\qty{1}{eV}}\right)
            }{
                \qty{2.5250e-18}{J}
            }
            Evaluating \autoref{eq:saha} with the above parameters then yields the ratio $n_\mathrm{e}^2/(n_0-n_\mathrm{e})$, which will be represented by $S$ for convenience:
            \calculation{\frac{n_\mathrm{e}^2}{n_0-n_\mathrm{e}} = S}{
                \frac{2(5.627)}{(\qty{6.0860e-10}{m})^3}\exp{\left(-\frac{(\qty{2.5250e-18}{J})}{(\evalkB)(\evalT)}\right)}
            }{
                \qty{2.5308e23}{m^{-3}}
            }
            The electron density can then be determined by solving  the quadratic equation:
            \begin{equation*}
                \frac{n_\mathrm{e}^2}{n_0-n_\mathrm{e}} = S \implies n_\mathrm{e}^2+Sn_\mathrm{e}-Sn_0 = 0
            \end{equation*}
            \calculation{n_\mathrm{e}}{
                \frac{-S+\sqrt{S^2-4(-Sn_0)}}{2}
            }{
                \qty{9.8612e23}{m^{-3}} = \qty{9.8612e17}{cm^{-3}}
            }

        \subsection{Absorption coefficient}
            For convenience, the IB absorption coefficient $\alpha$ formula is reproduced here, yielding a result in \unit{m^{-1}}:
            \begin{equation*}
                \ibalphaeq \tag{\ref{eq:ib_absorption} revisited}
            \end{equation*}
            It should be noted, again, that this form of the absorption coefficient expression is only valid with $n_\mathrm{e}$ in \unit{cm^{-3}} and $k_\mathrm{B}T_\mathrm{e}$ in \unit{eV} (i.e., \qty{1.293}{eV}). The Coulomb logarithm can be computed first---both approaches discussed in \autoref{sec:models_absorption} will be performed to compare their results. First is the following approximation, with $n_\mathrm{e}$ [\unit{cm^{-3}}] and $T_{E,\mathrm{e}}$ [eV]:
            \calculation{\ln{\Lambda}}{
                23 - \ln{(n_\mathrm{e}^{1/2}ZT_{E,\mathrm{e}}^{-3/2})}\\
                = 23 - \ln{((\qty{9.8612e17}{cm^{-3}})^{1/2}(1)(\qty{1.293}{eV})^{-3/2})}
            }{2.669}
            The alternate evaluation of the Coulomb logarithm is that of \textcite{johnstonCorrectValuesHighfrequency1973}:
            \begin{equation*}
                \ln{\Lambda} = \ln{\left(\frac{v_T}{\max{(\nu, \nu_\mathrm{p})}\rho_\mathrm{min}}\right)}
            \end{equation*}
            \calculation{v_T}{
                \sqrt{\frac{k_\mathrm{B}T}{m_\mathrm{e}}} = \sqrt{\frac{(\evalkB)(\qty{15000}{K})}{(\evalme)}}
            }{\qty{476800}{m.s^{-1}}}
            \autoref{fig:coulomb_freq} already showed that the laser frequency $\nu$ is much greater than the plasma frequency $\nu_\mathrm{p}$, but both will be compared here explicitly for completeness.
            \calculation{\nu}{\frac{c}{\lambda} = \frac{(\evalc)}{(\qty{1.07e-6}{m})}}{\qty{2.80e14}{s^{-1}}}
            \calculation{\nu_\mathrm{p}}{\frac{1}{2\pi}\sqrt{\frac{e^2}{\epsilon_0 m_\mathrm{e}}}\sqrt{n_\mathrm{e}} = (\qty{8978.85}{cm^{3/2}.s^{-1}})\sqrt{(\qty{9.8612e17}{cm^{-3}})}}{\qty{8.92e12}{s^{-1}}}
            \calculation{\max{(\nu, \nu_\mathrm{p})}}{\max((\qty{2.80e14}{s^{-1}}), (\qty{8.92e12}{s^{-1}}))}{\qty{2.80e14}{s^{-1}}}
            The minimum impact parameter $\rho_\mathrm{min}$ for electron--ion collisions is calculated as follows:
            \begin{equation*}
                \rho_\mathrm{min} = \max{\left(\frac{Ze^2}{k_\mathrm{B}T}, \frac{\hbar}{\sqrt{m_ek_\mathrm{B}T}}\right)}
            \end{equation*}
            \calculation{\frac{Ze^2}{k_\mathrm{B}T}}{
                \frac{(1)(\evale)^2}{(\evalkB)(\qty{15000}{K})}
            }{\qty{1.239e-19}{m}}
            \calculation{\frac{\hbar}{\sqrt{m_ek_\mathrm{B}T}}}{
                \frac{(\evalhbar)}{\sqrt{(\evalme)(\evalkB)(\qty{15000}{K})}}
            }{\qty{2.428e-10}{m}}
            \calculation{\rho_\mathrm{min}}{
                \max{\left(\frac{Ze^2}{k_\mathrm{B}T}, \frac{\hbar}{\sqrt{m_ek_\mathrm{B}T}}\right)} = \max{((\qty{1.239e-19}{m}),(\qty{2.428e-10}{m}))}
            }{\qty{2.428e-10}{m}}
            With this, the Coulomb logarithm evaluates to:
            \calculation{\ln{\Lambda}}{
                \ln{\left(\frac{v_T}{\max{(\nu, \nu_\mathrm{p})}\rho_\mathrm{min}}\right)} = \ln{\left(\frac{(\qty{476800}{m.s^{-1}})}{(\qty{2.80e14}{s^{-1}})(\qty{2.428e-10}{m})}\right)}
            }{1.948}
            Either approach for evaluating the logarithm can be taken. Although their result differs by 37\% in this case, the discrepancy decreases at higher temperatures, and the discrepancy's effect on the peak absorption coefficient is no greater than 10\% on both the temperature at which the peak occurs and the value of the peak. The absorption coefficient $\alpha$ can now be calculated:
            \begin{multline*}
                \alpha = \frac{7.8\times 10^{-7}Zn_\mathrm{e}^2\ln{\Lambda}}{\nu^2(k_\mathrm{B}T_\mathrm{e})^{3/2}} \left(1-\frac{\nu_\mathrm{p}^2}{\nu^2}\right)^{-1/2} \\
                = \frac{7.8\times 10^{-7}(1)(\qty{9.8612e17}{cm^{-3}})^2\ln{\Lambda}}{(\qty{2.80e14}{s^{-1}})^2(\qty{1.239}{eV})^{3/2}} \left(1-\frac{(\qty{8.92e12}{s^{-1}})^2}{(\qty{2.80e14}{s^{-1}})^2}\right)^{-1/2} \\
                = (\qty{7.019}{m^{-1}})\ln{\Lambda}\\
                \alpha = \biggl\{ \begin{array}{l}
                    \qty{18.7}{m^{-1}}\text{ for }\ln{\Lambda} = 2.669 \\
                    \qty{13.7}{m^{-1}}\text{ for }\ln{\Lambda} = 1.948 \\
                \end{array}
            \end{multline*}
