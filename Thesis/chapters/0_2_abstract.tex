\begin{plainchp}{Abstract}
    \addcontentsline{toc}{chapter}{Abstract}

    Recent advancements in space commercialization have made access to low Earth orbit (LEO) more affordable. However, for ambitious missions like rapid human transit to Mars, existing propulsion methods fall short. Laser-Thermal Propulsion (LTP) emerges as a promising solution, utilizing lasers to heat propellant gas and generate thrust with greater efficiency than traditional rocket engines. The objective of this thesis was to test a Version 1 (V1) lab-scale LTP thruster for interplanetary space flight, as well as design and test an improved Version 2 (V2) thruster. These will both be powered by a \qty{300}{W} Continuous Wave (CW), \qty{3}{kW} Quasi-Continuous Wave (QCW) \qty{1.07}{μm} fiber laser. Argon is used as the propellant at a pressure of \qty{20}{bar}. Spark initiation of QCW Laser-Supported Plasma (LSP) was successfully implemented in V1 and V2. \ce{NO2} seeding of the argon at partial pressures between \qtyrange{.12}{.55}{bar} showed the gas absorbed more than double the laser energy compared to pure argon propellant. A CW LSP was achieved in V2, lasting \qty{85.1}{ms}. This represents a 1.7 times longer lifetime than the maximum QCW pulse length of \qty{50.0}{ms} at this power. Average cold flow thrust of V2 was \qty{0.96}{N}. Parts to further improve V2 are presented to eventually enable a CW hot fire test of the thruster.

\end{plainchp}