\begin{plainchp}{Abstract}
    \addcontentsline{toc}{chapter}{Abstract}

    Recent advancements in space commercialization have made access to low Earth orbit (LEO) more affordable. However, for ambitious missions like rapid human transit to Mars, existing propulsion methods fall short. Laser-Thermal Propulsion (LTP) emerges as a promising solution, utilizing lasers to heat propellant gas and generate thrust with greater efficiency than traditional rocket engines. The objective of the research is to conduct a series of experiments with a lab-scale LTP thruster powered by a 1070 nm fiber laser. This research aims to measure critical performance parameters such as thrust and specific impulse, and to characterize the plasma heating core sustained by the laser. This project builds upon theoretical insights from prior studies and seeks to push the boundaries of experimental research into this propulsion system. The laser will generate a plasma which will be thoroughly analyzed using a variety of diagnostic instruments. The initial phase of the research has successfully demonstrated reliable ignition of argon plasma, setting the stage for further experimentation. This began with pulsed operation of a first-generation thruster. Subsequent phases involved the design of a second-generation thruster, and testing under continuous operation.

\end{plainchp}