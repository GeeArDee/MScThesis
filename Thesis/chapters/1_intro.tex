\chapter{Introduction} \label{chp:intro}
    % AIAA Talk written out
    In 2016, the Breakthrough Starshot initiative was proposed based on the work of Professor Philip Lubin at UC Santa Barbara. This mission involves beaming \qty{1}{g} space probes to \qty{20}{\%} the speed of light, using ground-based laser arrays.

    This could be enabled by a Moore's law in fiber laser technology, with a rapid doubling of power and a similar exponential decrease in costs. The era of giant phased-array lasers is coming! As a near-term stepping stone using a smaller array, we can couple the laser to a gas, reducing efficiency but increasing thrust. This would allow rapid interplanetary transfers and is called Laser-Thermal Propulsion (LTP).

    The system architecture is similar to Breakthrough Starshot: a \qty{10}{m} laser array beams \qty{100}{MW} of power to an orbiting spacecraft for injection burns. Our team proposed a conceptual design of an LTP spacecraft in 2022 in Acta Astronautica [CITE]. With a \qty{1000}{kg} payload, about \qty{6}{kN} of thrust and \qty{3000}{s} of $I_{sp}$, we do a \qty{1}{h} maneuver to get \qty{14}{km/s} of delta-V and reach Mars in 45 days.

    In the thrust chamber of the vehicle, hydrogen is introduced. The laser is focused inside, creating a plasma core. Colder hydrogen flows around it and is heated by the plasma to \qty{10000}{K}. It is then exhausted though a conventional converging-diverging nozzle, imparting thrust. My research concentrates on the laser-plasma coupling, and the heating of the bulk gas.

    We've identified 3 main problems. Initiation of laser ionization, propagation of the plasma into the mixture and heating of the gas. Our experimental approach, which I will go into more detail point-by-point later, is to visualize the LSP wave, investigate the power/pressure threshold, measure the power absorption and the pressure rise of the gas.

    Conceptually, laser-thermal propulsion started in the 1970s with Raizer and Kantrowitz. Keefer and Toyoda were among the inspiration for our apparatus. Our contribution is to use fiber lasers instead of CO2 lasers. Fiber lasers give more range for the same aperture size due to laser wavelength. This is also the laser type that Breakthrough Starshot uses. 
    
    Now, to our experiment. A collimator in blue, here, beams the laser through a lens into our pressurized test section. The beam focuses onto our ignition device. We have mostly used a tungsten wire as a target, and have started laser supported plasmas with an electric arc as well. The test section is pressurized up to 20 bar of Argon, as it is inert and easy to ionize. As for diagnostics, we have a spectrometer, a camera, a pressure transducer and a power meter. This schematic will appear in future slides for reference. We use an IPG photonics Ytterbium fiber laser. Wavelength is 1 micron, CW power of 300 W and can go up to 3kW when pulsed for 10 ms.

    Here are the results of our experiments. We were able to generate LSP in argon. The laser pulse is 10 ms, after which the plasma dissipates. The plasma starts at the wire, then grows at 9 m/s towards the laser source. LSP length is about 20 mm. We studied the power/pressure threshold at which the plasma is stable in argon. In the x-axis we have the pressure while in the y-axis we have laser power. The colored dots are our own data, compared to 3 sources from literature: Zimakov, Matsui and Lu. We have good agreement with the previous measurements, and the trends agree. In our case, arc ignition is not as efficient as wire ignition. To determine these points, our laser shots started at 3 kW to ignite the plasma, then the power was stepped down. If the plasma was stable, we recorded the power/pressure data.

    For our laser to plasma energy absorption, we measured residual power with a Gentec-EO power meter. This gave an efficiency of 80\%. Now, for the heat deposition, we measured the pressure rise in the bulk gas. The x-axis is time and the y is our delta p. In the red zone, the laser is on. Knowing the argon mass, we determined the heat deposition. This gave us an efficiency of approximately 15\%, averaging across many shots.

    We completed flowing tests with a choked orifice nozzle. We removed the power meter. As expected, the higher the flow rate, the lower the delta p.

    In conclusion, we used a 3kW fiber laser to generate laser-sustained plasma. Our power-pressure curve agrees with literature and we have 80\% laser to plasma absorption. 15\% of the laser power is deposited in the gas as heat and flowing thrust measurements are our next step.