\begin{plainchp}{Abstract}
    \addcontentsline{toc}{chapter}{Abstract}
    
    For ambitious space missions like rapid human transit to Mars, conventional propulsion methods fall short. Laser-Thermal Propulsion (LTP) utilizes lasers to heat propellant gas, generating thrust with potentially greater specific impulse than traditional rocket engines. Two lab-scale LTP thrusters were tested in this thesis, denoted Version 1 (V1) and Version 2 (V2). Version 1 permitted initial testing and visualization of the Laser-Sustained Plasma (LSP) wave propagation. As a prototype for an actual thruster, Version 2 was optimized for thrust measurements. The design process of the improved V2 thruster is also presented. These thrusters will both be powered by a \qty{300}{W} Continuous Wave (CW), \qty{3}{kW} Quasi-Continuous Wave (QCW) \qty{1.07}{μm} fiber laser. Argon was used as the propellant at a pressure of \qty{20}{bar}, selected for its ease of ionization. Using an automotive type coil, spark initiation of QCW Laser-Supported Plasma was successfully implemented in V1 and V2. Seeding of the argon with nitrogen dioxide (\ce{NO2}) at partial pressures between \qtyrange{.12}{.55}{bar} showed the gas absorbed more than double the laser energy compared to pure argon propellant. To increase laser flux to the plasma, an optical system of two lenses was designed. Different lenses were compared using ray tracing software (WinLens3D). A Continuous Wave (CW) LSP was achieved in V2, lasting \qty{85.1}{ms}. This represents a 1.7 times longer lifetime than the maximum QCW pulse length of \qty{50.0}{ms} at this power. Average cold flow thrust of V2 was \qty{0.96}{N}. In order to interpret the experimental results, a zero-dimensional (0D) heat transfer model was written in Python, using Bremsstrahlung as the mechanism of radiation. Finally, paths to further improve the V2 thruster and thrust stand are presented to eventually enable a CW hot fire test.

\end{plainchp}