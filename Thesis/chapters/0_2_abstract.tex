\begin{plainchp}{Abstract}
    \addcontentsline{toc}{chapter}{Abstract}
    
    Laser-Thermal Propulsion (LTP) utilizes lasers to heat propellant gas, generating thrust with potentially greater specific impulse than traditional rocket engines. In this thesis, two lab-scale LTP thrusters were tested: Version 1 (V1) and Version 2 (V2). The design process of the improved V2 thruster is also presented. These will both be powered by a \qty{300}{W} Continuous Wave (CW), \qty{3}{kW} Quasi-Continuous Wave (QCW) \qty{1.07}{μm} fiber laser. Argon is used as the propellant at a pressure of \qty{20}{bar}. Spark initiation of QCW Laser-Supported Plasma (LSP) was successfully implemented in V1 and V2. \ce{NO2} seeding of the argon at partial pressures between \qtyrange{.12}{.55}{bar} showed the gas absorbed more than double the laser energy compared to pure argon propellant. A CW LSP was achieved in V2, lasting \qty{85.1}{ms}. This represents a 1.7 times longer lifetime than the maximum QCW pulse length of \qty{50.0}{ms} at this power. Average cold flow thrust of V2 was \qty{0.96}{N}. In order to interpret the experimental results, a zero-dimensional (0D) heat transfer model was written in Python, using Bremsstrahlung as the mechanism of radiation. Finally, parts to further improve the V2 thruster are presented to eventually enable a CW hot fire test.

\end{plainchp}