\chapter{Facility design}

    \section{Laser}

        Use section like this?

    \section{Version 1} \label{sec:design_v1}

        The design process of the first generation thruster, called Version 1 (V1), can be found in \textcite{duplayArgonLaserPlasmaThruster2024a}. It proved to be a dependable prototype, repurposed from a previous unrelated experiment. However, it presented problems that required a second generation prototype to be designed and manufactured.

        [Add short description of V1 here]

        V1 was powered by an IPG Photonics YLR-300/3000-QCW-MM-AC Ytterbium fiber laser, 

        \begin{figure}[!ht]
            \centering
            \begin{subfigure}[t]{\textwidth}
                \includegraphics[width=0.85\textwidth]{assets/3 design/finalsetup_static.pdf}
                \caption{Static setup}
            \end{subfigure}
            \hfill
            \begin{subfigure}[t]{\textwidth}
                \includegraphics[width=0.85\textwidth]{assets/3 design/finalsetup_flowing.pdf}
                \caption{Flowing setup}
            \end{subfigure}
            \caption{V1 LTP thruster}
            \label{fig:V1 setup}
        \end{figure}

        \subsection{Capabilities}

            298 recorded pulsed laser shots were conducted with V1, exploring the power-pressure threshold, wire initiation and spark initiation.

        \subsection{Issues}

            Too heavy -> too much friction on rail for thrust tests. This was mitigated in part by a rope system.

            Critically, rubber seals were exposed to laser path during continuous lasing with a lower focal length lens (picture), severely burning them in the only CW test conducted. A shorter test section designed for a \qty{100}{mm} focal length lens would solve this.







    \section{Version 2} \label{sec:design_v2}

        To improve upon the V1 facility, an entire redesign was done. This resulted in the much smaller Version 2 (V2) purpose-built LTP thruster at the end of April 2024.

        \subsection{Requirements}

            The following requirements were developed for the design of the V2 thruster. The objective was to detect a measurable difference in thrust between an argon cold gas thruster and an argon “hot gas” thruster, heated by a laser supported plasma (LSP).

            \begin{enumerate}
                \item Laser thruster
                \begin{enumerate}
                    \item A \qty{300}{W} Continuous Wave (CW) \qty{1070}{nm} laser shall sustain the plasma (Nominal power \qty{300}{W}, actual max power \qty{350}{W})
                    \item The thruster shall have a minimum safe “hot” operation time of \qty{30}{s}
                    \begin{enumerate}
                        \item In the event of failed LSP initiation, the thruster shall safely absorb the total laser power for at least \qty{10}{s}
                    \end{enumerate}
                    \item An optical path shall be present to let the laser into the thruster, utilizing a \qty{100}{mm} focal length lens at minimum and a collimated beam with a maximum diameter of \qty{30}{mm}
                    \begin{enumerate}
                        \item The optical components shall not be damaged by the laser flux
                    \end{enumerate}
                    \item Argon shall be used as the working fluid
                    \begin{enumerate}
                        \item The argon feed gas shall be at room temperature
                    \end{enumerate}
                    \item A gas feed path shall bring argon gas into the thruster
                    \begin{enumerate}
                        \item The gas feed shall be choked at the thruster inlet
                        \item The gas feed shall be evenly distributed in the thruster
                    \end{enumerate}
                    \item The mass flow rate of the argon gas shall be measured and controlled by interchangeable upstream choked orifices
                    \item The maximum allowable operating pressure (MAOP) of the thruster shall be 50 bar
                    \begin{enumerate}
                        \item The nominal pressure of the thruster shall be 25 bar
                    \end{enumerate}
                    \item A converging-diverging exhaust nozzle shall be designed to accelerate the gas to a supersonic speed
                    \begin{enumerate}
                        \item The nozzle shall be easily changeable
                    \end{enumerate}
                    \item A 1/8" NPT port for a pressure transducer shall be present along the thruster
                    \item An optical port shall be present for spectrometry measurements of the plasma
                    \item The thruster shall be installed on a thrust stand (See section 3. Thrust stand)
                \end{enumerate}
                \item Initiation system/electrical
                \begin{enumerate}
                    \item The LSP shall be ignited by an electrical spark
                    \item The spark gap shall be measurable, controllable, and repeatable
                    \item The spark shall be generated by an AEM 30-2853 High Output Smart Coil, supplying \qty{41}{kV} with up to \qty{118}{mJ}
                    \item All parts of the thruster and thrust stand shall be directly or indirectly connected to a common electrical ground
                \end{enumerate}
                \item Thrust stand
                \begin{enumerate}
                    \item The thrust stand shall measure thrust on the order of \qtyrange{0.1}{5}{N}
                    \item The thrust stand shall minimize friction losses
                    \item The thrust stand shall be securely fixed using standard optical breadboard mounting hardware
                \end{enumerate}
            \end{enumerate}

            With these requirements, preliminary geometric dimensions of the V2 thruster could commence. It was expected to be much smaller than V1, as the goal was to isolate the LSP region and increase heat flux to the gas.

        \subsection{Sizing the double choked LTP thruster} 

            When adding energy to the thruster chamber with a laser, it is useful to choke the inflow upstream of the chamber. Indeed, this keeps the $P_0$ and $\dot{m}_\mathrm{in}$ constant, so the increase in chamber pressure can be interpreted as a measure of energy deposition (as was discussed in \autoref{chp:models}). The second choke happens at the nozzle to accelerate the hot gas to a supersonic speed. Therefore, this configuration is double choked, a classic problem in compressible fluid mechanics.
            
            The starting assumptions were the following: a \qty{300}{W} power input (the laser) supplies energy to an LTP experiment that has an internal pressure of \qty{25}{bar}, with a \qty{50}{bar} feed pressure. It is required that the hot gas operation (laser on) increases the gas' exit velocity to twice that of the cold gas operation (laser off). We will determine the gas mass flow rate and the diameter of the two orifices needed to choke the flow.

            \begin{figure}[h]
                \centering
                \includegraphics[width=0.8\linewidth]{assets/3 design/Double choked LTP thruster.pdf}
                \caption{Cutaway of a double choked LTP thruster showing both choking orifices: the metering valve and the nozzle.}
                \label{fig:double choke sizing}
            \end{figure}
            
            Starting with a cold gas thruster using argon, the speed of sound ($c_0$) is \qty{323}{m/s}. This is at ambient temperature (\qty{300}{K}), as we have no laser energy to heat the gas in this case. With a nozzle, the gas is accelerated to approximately twice this speed. The $v_\mathrm{exit}$, which is our main performance parameter, is therefore \qty{646}{m/s}.
            
            Laser on (hot) operation will now be examined. Taking the previous $v_\mathrm{exit}$ and ionizing the whole flow, it is posed that our efficiency is doubled. This gives a $v_\mathrm{exit}\approx \qty{1300}{m/s}$. What nozzle throat size is necessary for this $\dot m$ with $p_\mathrm{chamber} = \qty{25}{bar}$? We know that $\mathrm{MW_{Ar}} = \qty{40}{g/mol}$. The speed of sound is $c = \sqrt{\gamma R T}$. As we want to double the speed of sound, we are multiplying the temperature by 4.
            \[\text{Power} = \dot m (h_2-h_1)
            = \dot m c_p (T_2-T_1)\]
            Using a constant $c_p$ of argon of \qty{0.520}{kJ.kg^{-1}.K^{-1}}, the calculated $\dot m$ is \qty{0.641}{g/s}.
            
            Fliegner's formula describes the mass flow rate of an isentropic flow:
            \[\frac{\dot m}{A} = p_0\sqrt{\frac{\gamma}{T_0 R}}\frac{M}{(1+\frac{\gamma-1}{2}M^2)^{(\frac{\gamma+1}{2(\gamma-1)})}}\]
            With $\gamma = \frac{c_p}{c_v} = 1.666$ for argon and choked flow at the nozzle, the area and the diameter of the circular nozzle are \qty{0.176}{mm^2} and \qty{0.473}{mm}, respectively. These calculations can be repeated for the feed orifice, with the same $\dot m$, a pressure of \qty{50}{bar} and ambient temperature. This gives us an orifice size of about 0.2 mm.
        
        \subsection{Additional V2 systems}
            
            Static pressure testing up to \qty{75}{bar} for \qty{25}{minutes} was completed successfully [HOWEVER]

            \subsubsection{Electrodes}
                The \qty{44}{kV} wire having burst during pressure tests, different electrodes were necessary. Molded dielectric epoxy (Stycast ES 1001 [LINK to website]) around an industrial sewing needle core was chosen, as it was economical and the outer diameter of the electrodes could be precisely controlled by sanding the surface of the set epoxy. Molds were 3d printed and Mann Ease Release\texttrademark 300 was applied to all their inside surfaces.

                \begin{figure}[!ht]
                    \centering
                    \begin{subfigure}[t]{0.30\textwidth}
                        \centering
                        \includegraphics[width=\textwidth]{assets/3 design/Molds.jpg}
                        \caption{Molds with steel needle core in place}
                    \end{subfigure}
                    \hfill
                    \begin{subfigure}[t]{0.30\textwidth}
                        \centering
                        \includegraphics[width=\textwidth]{assets/3 design/Mold process.jpg}
                        \caption{Molding process}
                    \end{subfigure}
                    \hfill
                    \begin{subfigure}[t]{0.30\textwidth}
                        \centering
                        \includegraphics[width=\textwidth]{assets/3 design/V2 electrodes.jpg}
                        \caption{Assembled electrodes with Ultra-Torr cap and electrical connectors}
                        \label{fig:Assembled electrode}
                    \end{subfigure}

                    \caption{Electrode manufacturing process}
                \end{figure}

                The electrodes were then sanded down to fit tightly into Ultra-Torr vacuum connectors [link to swagelok and part number]. Although these connectors were not designed for high pressure, previous experience has shown that they are appropriate up to about \qty{20}{bar} of internal pressure if tightened enough.

                The result is presented in \autoref{fig:Assembled electrode}.

                Once installed in the V2 thruster, the electrodes were pushed into contact with each other and the Ultra-Torr connectors tightened. Statically pressurizing V2 to \qty{20}{bar} was enough to slightly separate the electrodes from one another. [add text for flow testing?]
    
    \section{Optics design}
                
                Due to the low continuous laser power compared to experiments in the literature, increasing the laser flux with small a focus is critical.

                Two ways can be used to calculate the spot size of a laser. Firstly, ray tracing software such as WinLens calculate the geometry of paraxial \footnote{Rays having small angles and distances to the optical axis} rays and show the path of these rays at the focus. Secondly, this equation \cite{LaserSpotSize} can be used: (ideal?)
                
                \[
                \text{Spot size}(mm) = \frac{4 \times \text{Focal length}(mm) \times \text{Wavelength}(mm) \times M^2}{\pi \times \text{Beam diameter at lens}(mm)}
                \]

                The beam propagation factor $M^2$ is a scale to measure beam quality. A diffraction-limited Gaussian beam has the minimum $M^2$ of 1 \cite{hechtUnderstandingLasersEntry2019}. 
                
                The YLR-300/3000 laser has a BPP of \qty{2}{mm.mrad} [Refer to appendix laser's data sheet].

                
                
                It is also implemented in WinLens, but is not based on ray tracing.

                The following spot diagrams were produced with WinLens. Note the difference in scales and spacing used throughout.

            

                \begin{table}[!ht]
                    \centering
                    \caption{Simulated focal length and spot diameter of various lens assemblies in WinLens3D. Laser flux is also calculated for \qty{300}{W} of incident power}
                    \label{tab:laser flux}
                    \begin{tabularx}{\textwidth}{@{}lX<{\raggedright}X<{\raggedright}X<{\raggedright}X<{\raggedright}@{}}
                    \toprule
                    Lens & Nominal focal length (\unit{mm}) & Focal length at \qty{1070}{nm} (\unit{mm})& Beam diameter at focus (\unit{mm}) & Laser flux at \qty{300}{W} (\unit{W/cm^2}) \\ \midrule
                    Single & 125           &  122   &    0.0863 ?  0.20?     &  \\
                    Single & 100           &  93   &    0.20   &  \\
                    Double & 500, 150      &  110    &    0.08   &  \\
                    \bottomrule
                    \end{tabularx}
                \end{table}

                (Maybe compare with minimum maintenance intensity of previous literature)

                In a two element system, the longest focal length lens should be placed first, as the diameter of the beam entering the second lens is maximized. However, it is placed after here because it is impossible to mount before. The difference in spot size is acceptable. [maybe compare?]

                Practical considerations for experiments are the following. When aligning the laser with the red visible guide beam, chromatic abberations [\dots] \cite{hechtUnderstandingLasersEntry2019}.

                The LSP will also be formed upstream of the laser focus when there is no gas flow. Therefore, the focus needs to be slightly after the ignition system. [does this change with flowing LSP?]

            \subsection{V1 Bringing the pulsed power down and optical design} \label{sec:pulse_power_down_V1}
            
                Pulsed shots at lower power levels revealed a difficulty to initiate the LSP below \qty{20}{\%} power, which corresponds to about \qty{620}{W}. This poses a problem, as the maximum CW power of the laser is significantly lower, at \qty{350}{W}. A test campaign was started in February 2024 to determine if LSP initiation in the V1 thruster was possible under this maximum CW power level.
                
                To obtain LSP initiation, a high enough laser flux is needed. With a fixed power, it is necessary to focus the laser down to the smallest area possible to get the highest flux. Quantifying the diameter of this focus was therefore the first step. 
    
                %Section on quantifying diameter, laser optics basics (email from thorlabs guy)
    
                For a multi-element system, the spot diameter must be calculated numerically with ray tracing software. WinLens 3D Basic \cite{QioptiqQShopFree} was used here, as it is free and powerful enough for this application. The single element system was also simulated in this software to verify the calculations.
    
                Now that the diameter of the focus is known, two avenues are possible to improve it: a shorter focal length or a multi-lens system \cite{LensTutorial}. At first, a single lens with a \qty{125}{mm} focal length (Thorlabs LA1384-C [CITE 125mm lens]) was used, as it was the simpler option. During these shots, the goal was to achieve LSP initiation at or below \qty{11}{\%} pulsed power, or \qty{340}{W}. The following graph shows LSP initiation attempts at various power settings and axial lens positions. 20 pulsed laser shots were performed for each point on the graph. If at least one was successful at igniting LSP, it was recorded as such. This graph can also be interpreted as a beam profiling for LSP conditions.
                
                % Graph of 125mm lens
                \begin{figure}[!ht]
                    \centering
                    \includegraphics[width=\textwidth]{assets/4 experiments/125lens.pdf}
                    \caption{125 mm focal length lens}
                \end{figure}
    
                \begin{figure}[!ht]
                    \centering
                    \includegraphics[width=0.75\textwidth]{assets/4 experiments/125mm_focus_threshold.pdf}
                    \caption{125 mm focal length lens}
                \end{figure}
                
                Initiation at \qty{11}{\%} was attained once, but it was not possible to replicate this. A tighter focus was necessary to increase initiation reliability.
    
                % Graph of multi-lens
                \begin{figure}[!ht]
                    \centering
                    \includegraphics[width=\textwidth]{assets/4 experiments/500 and 150 lenses.pdf}
                    \caption{Multi-lens system}
                \end{figure}
    
    
                \begin{figure}[!ht]
                    \centering
                    \includegraphics[width=0.75\textwidth]{assets/4 experiments/duallens_focus_threshold.pdf}
                    \caption{Multi-lens system}
                \end{figure}
    
                % Section on power meter reading lower pulsed power at these low power settings, like about 200W
    
                The completion of these tests validated LSP generation in the CW power regime of the laser. The V2 thruster was then set up to test CW operation with flowing argon.
            
            \subsection{V2 Bringing the pulsed power down, again}
            
                To prevent the damage to the thruster seen previously, a rear window mount was manufactured. This allows the laser energy that is not absorbed to pass freely through the apparatus, also enabling power meter measurements. 
                
                % Photo of rear window mount and damaged window
    
                As can be seen above, the window suffered laser damage after this round of testing. However, this damage proved minor as the window was used to align the laser focus with the spark gap, not to measure power absorption.
                
                The following figure presents the LSP initiation attempts with focus distance, similarly to the graphs presented in \autoref{sec:pulse_power_down_V1}.
    
                \begin{figure}[!ht]
                    \centering
                    \includegraphics[width=0.75\textwidth]{assets/4 experiments/V2_focus_threshold.pdf}
                    \caption{LSP threshold graph for V2}
                \end{figure}
    
                The real amount of power in the pulsed shots was also measured to validate the \qty{11}{\%} threshold quoted previously. 10 shots each at \qtylist{10; 12}{\%} were measured with the power meter, with statistics compiled by the power meter software (see \autoref{label})
    
                \begin{table}[!ht]
                    \caption{Statistics from the power meter after 10 times \qty{50}{ms} laser shots at \qty{10}{\%} and \qty{12}{\%} power}
                    \label{tab:laser shot statistics}
                    \begin{tabular}{lll}
                    \textbf{Value {[}Unit{]}} & \textbf{10x 50 ms shots at 10\% power} & \textbf{10x 50 ms shots at 12\% power} \\ \hline
                    Average value {[}J{]}  & 9.985 & 12.89 \\
                    Maximum value {[}J{]}  & 10.2  & 13.3  \\
                    Minimum value {[}J{]}  & 9.63  & 12.0  \\
                    RMS Stability {[}\%{]} & 1.690 & 2.811 \\
                    PTP Stability {[}\%{]} & 5.599 & 10.31 \\
                    Average power {[}W{]}  & 0.490 & 1.14  \\
                    Std deviation {[}J{]}  & 0.169 & 0.362 \\ \hline
                    \end{tabular}
                \end{table}
                
                At 10\% power, an 9.985 J average during \qty{50}{ms} gives an average power of \qty{200}{W}, lower than the expected \qty{300}{W} For \qty{12}{\%}. This revealed a higher power threshold, in terms of percentage, than previously thought. Extrapolating from these measurements, \qty{300}{W} is achieved at \qty{13.5}{\%}. This validated that the 