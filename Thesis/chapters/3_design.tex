\chapter{Facility design}

    \section{Version 1 test section} \label{sec:design_v1}

        The design process of the first generation thruster, called Version 1 (V1), can be found in \textcite{duplayArgonLaserPlasmaThruster2024a}. It proved to be a dependable prototype, repurposed from a previous unrelated experiment. However, it presented problems that required a second generation prototype to be designed and manufactured.

        \begin{figure}[!ht]
            \centering
            \begin{subfigure}[t]{\textwidth}
                \includegraphics[width=0.85\textwidth]{assets/3 design/finalsetup_static.pdf}
                \caption{Static setup}
            \end{subfigure}
            \hfill
            \begin{subfigure}[t]{\textwidth}
                \includegraphics[width=0.85\textwidth]{assets/3 design/finalsetup_flowing.pdf}
                \caption{Flowing setup}
            \end{subfigure}
            \caption{V1 LTP thruster}
            \label{fig:V1 setup}
        \end{figure}

            298 recorded pulsed laser shots were conducted with V1, exploring the power-pressure threshold, wire initiation and spark initiation. A side window permitted direct visualization of the LSP with a high speed camera (Photron )

            However, this test section was made of steel, creating too much friction on its rails during thrust tests. This was mitigated in part by a rope system mentioned in \textcite{duplayArgonLaserPlasmaThruster2024a}, but was not found to be repeatable. Having the LSP heat a smaller internal volume than the \qty{0.4}{L} of V1 was also desirable, as a greater effect on internal pressure and thrust would be seen.

            Critically, rubber seals were exposed to the laser path during continuous (CW) lasing with a lower focal length lens (picture), severely burning them in the only CW test conducted. A shorter test section designed for a \qty{100}{mm} focal length lens would solve this.

    \section{Version 2} \label{sec:design_v2}

        To improve upon the V1 facility, an entire LTP thruster redesign was done. This resulted in the much smaller Version 2 (V2) purpose-built LTP thruster at the end of April 2024.

        \subsection{Requirements}

            The following requirements were developed for the design of the V2 thruster. The objective was to detect a measurable difference in thrust between an argon cold gas thruster and an argon “hot gas” thruster, heated by a laser supported plasma (LSP).

            \begin{enumerate}
                \item Laser thruster
                \begin{enumerate}
                    \item A \qty{300}{W} Continuous Wave (CW) \qty{1070}{nm} laser shall sustain the plasma (Nominal power \qty{300}{W}, actual max power \qty{350}{W})
                    \item The thruster shall have a minimum safe “hot” operation time of \qty{30}{s}
                    \begin{enumerate}
                        \item In the event of failed LSP initiation, the thruster shall safely absorb the total laser power for at least \qty{10}{s}
                    \end{enumerate}
                    \item An optical path shall be present to let the laser into the thruster, utilizing a \qty{100}{mm} focal length lens at minimum and a collimated beam with a maximum diameter of \qty{30}{mm}
                    \begin{enumerate}
                        \item The optical components shall not be damaged by the laser flux
                    \end{enumerate}
                    \item Argon shall be used as the working fluid
                    \begin{enumerate}
                        \item The argon feed gas shall be at room temperature
                    \end{enumerate}
                    \item A gas feed path shall bring argon gas into the thruster
                    \begin{enumerate}
                        \item The gas feed shall be choked at the thruster inlet
                        \item The gas feed shall be evenly distributed in the thruster
                    \end{enumerate}
                    \item The mass flow rate of the argon gas shall be measured and controlled by interchangeable upstream choked orifices
                    \item The maximum allowable operating pressure (MAOP) of the thruster shall be 50 bar
                    \begin{enumerate}
                        \item The nominal pressure of the thruster shall be 25 bar
                    \end{enumerate}
                    \item A converging-diverging exhaust nozzle shall be designed to accelerate the gas to a supersonic speed
                    \begin{enumerate}
                        \item The nozzle shall be easily changeable
                    \end{enumerate}
                    \item A 1/8" NPT port for a pressure transducer shall be present along the thruster
                    \item An optical port shall be present for spectrometry measurements of the plasma
                    \item The thruster shall be installed on a thrust stand (See section 3. Thrust stand)
                \end{enumerate}
                \item Initiation system/electrical
                \begin{enumerate}
                    \item The LSP shall be ignited by an electrical spark
                    \item The spark gap shall be measurable, controllable, and repeatable
                    \item The spark shall be generated by an AEM 30-2853 High Output Smart Coil, supplying \qty{41}{kV} with up to \qty{118}{mJ}
                    \item All parts of the thruster and thrust stand shall be directly or indirectly connected to a common electrical ground
                \end{enumerate}
                \item Thrust stand
                \begin{enumerate}
                    \item The thrust stand shall measure thrust on the order of \qtyrange{0.1}{5}{N}
                    \item The thrust stand shall minimize friction losses
                    \item The thrust stand shall be securely fixed using standard optical breadboard mounting hardware
                \end{enumerate}
            \end{enumerate}

            With these requirements, preliminary geometric dimensions of the V2 thruster could commence. It was expected to be much smaller than V1, as the goal was to isolate the LSP region and increase heat flux to the gas.
        
        \subsection{Test section and thrust stand}

            \begin{figure}[!ht]
                \centering
                \begin{subfigure}[t]{0.45\textwidth}
                    \centering
                    \includegraphics[width=\textwidth]{assets/3 design/V2 Static configuration.jpg}
                    \caption{Static configuration. Note the extension part and window mount.}
                \end{subfigure}
                \hfill
                \begin{subfigure}[t]{0.45\textwidth}
                    \centering
                    \includegraphics[width=\textwidth]{assets/3 design/V2 flowing setup.jpg}
                    \caption{Flowing configuration. The nozzle is held by the rear plate.}
                \end{subfigure}
                \caption{V2 LTP thruster}
                \label{fig:V2 setup}
            \end{figure}

            The thrust stand is a ball bearing carriage (McMaster-Carr 6709K12) mounted on a \qty{15}{mm} wide rail (McMaster-Carr 6709K33). A string through a pulley holds a variable weight, adding a preload to the test section. This ensures adequate contact between the test section and the load cell. Two load cells are used with different force sensing range: Honeywell FSG020WNPB (\qtyrange{0}{20}{N}) and Honeywell FSG005WNPB (\qtyrange{0}{5}{N}).

        \subsection{Laser}

            The laser used as the plasma's power source is an IPG Photonics YLR-300/3000-QCW-MM-AC Ytterbium fiber laser. The wavelength of the emitted light is \qty{1070}{nm}. Its nominal maximum power is \qty{3}{kW} quasi-continuous wave (QCW) or \qty{300}{W} continuous wave (CW). At \qty{3}{kW}, a QCW pulse has a maximum duration of \qty{10}{ms}. The maximum duration of a \qty{300}{W} QCW pulse is \qty{50}{ms} The laser light exits through an IPG Photonics P30-001736 collimator. The output beam is \qty{30}{mm} in diameter.

            \begin{figure}[!ht]
                \centering
                \begin{subfigure}[t]{0.45\textwidth}
                    \centering
                    \includegraphics[width=\textwidth]{assets/3 design/Laser box.jpg}
                    \caption{IPG Photonics YLR-300/3000-QCW-MM-AC laser}
                \end{subfigure}
                \hfill
                \begin{subfigure}[t]{0.45\textwidth}
                    \centering
                    \includegraphics[width=\textwidth]{assets/3 design/Laser aperture.jpg}
                    \caption{IPG Photonics P30-001736 collimator}
                \end{subfigure}
                \caption{Laser system}
            \end{figure}

            Calibration reports for the laser and the collimator can be found in the appendix. [ADD TO APPENDIX]

        \subsection{Spark initiation system}

            AEM coil and electronics box.

        \subsection{Timing control}

            Correct timing of the laser and spark initiation is necessary to initiate LSP when the laser is in QCW mode, and to minimize damage to V2's nozzle in CW mode. To this end, delay generators are used (BNC models 7010 and 7055).

        \subsection{Data acquisition (DAQ) system and oscilloscope}

            Load cell and pressure transducer voltage is sent to a DATAQ Instruments DI-2018. This data is streamed to a personal computer by USB, where the thrust and pressure traces can be saved for analysis. Two pressure sensors were used: [PCB and OMEGA]

            \begin{figure}[!ht]
                \centering
                \includegraphics[width=0.50\textwidth]{assets/3 design/DAQ electronics.jpg}
                \caption{DAQ system}
                \label{fig:DAQ}
            \end{figure}

            [Photos of pressure sensor and PCB assembly]

        \subsection{High speed camera}

            A Photron SA5 high speed camera was used on certain LSP shots to validate 
            
            Due to the fact that no side window was present on V2, the Photron SA 5 high speed camera looking into the core of the thruster was used to validate.

            [Photo of Photron looking into thruster]