\chapter{Experiments}

    The following chapter will explain the experiments undertaken with both V1 and V2 thrusters, including the improvements made to V2.

    \section{V1 expriments}

        \subsection{Spark ignition}
            
            As was discussed in \textcite{duplayArgonLaserPlasmaThruster2024a}, spark ignition was not reliable enough with the available electrode system. 

        \subsection{Bringing the pulsed power down and Optical design (move to design)} \label{sec:pulse_power_down_V1}
            
            Pulsed shots at lower power levels revealed a difficulty to ignite the LSP below \qty{20}{\%} power, which corresponds to about \qty{620}{W}. This poses a problem, as the maximum CW power of the laser is significantly lower, at \qty{350}{W}. A test campaign was started in February 2024 to determine if LSP ignition in the V1 thruster was possible under this maximum CW power level.
            
            To obtain LSP ignition, a high enough laser flux is needed. With a fixed power, it is necessary to focus the laser down to the smallest area possible to get the highest flux. Quantifying the diameter of this focus was therefore the first step. 

            %Section on quantifying diameter, laser optics basics (email from thorlabs guy)

            For a multi-element system, the spot diameter must be calculated numerically with ray tracing software. WinLens 3D Basic \cite{winlens} was used here, as it is free and powerful enough for this application. The single element system was also simulated in this software to verify the calculations.

            Now that the diameter of the focus is known, two avenues are possible to improve it: a shorter focal length or a multi-lens system \cite{thorlabs}. At first, a single lens with a \qty{125}{mm} focal length (Thorlabs LA1384-C \cite{125mm lens}) was used, as it was the simpler option. During these shots, the goal was to achieve LSP ignition at or below \qty{11}{\%} pulsed power, or \qty{340}{W}. The following graph shows LSP ignition attempts at various power settings and axial lens positions. 20 pulsed laser shots were performed for each point on the graph. If at least one was successful at igniting LSP, it was recorded as such. This graph can also be interpreted as a beam profiling for LSP conditions.
            
            % Graph of 125mm lens
            \includegraphics[width=\textwidth]{assets/5 results/125lens.pdf}

            \begin{figure}[h]
                \centering
                \includegraphics[width=0.75\textwidth]{assets/5 results/125mm_focus_threshold.pdf}
                \caption{125 mm focal length lens}
            \end{figure}
            
            Ignition at \qty{11}{\%} was attained once, but it was not possible to replicate this. A tighter focus was necessary to increase ignition reliability.

            % Graph of multi-lens
            \includegraphics[width=\textwidth]{assets/5 results/500 and 150 lenses.pdf}

            \begin{figure}[h]
                \centering
                \includegraphics[width=0.75\textwidth]{assets/5 results/duallens_focus_threshold.pdf}
                \caption{Multi-lens system}
            \end{figure}

            % Section on power meter reading lower pulsed power at these low power settings, like about 200W

            The completion of these tests validated LSP generation in the CW power regime of the laser. The V2 thruster was then set up to ultimately test CW operation with flowing argon.
        
        \subsection{\ce{NO2} seeding}
            
            As the plasma emits in the ultraviolet (UV) range, it is necessary to seed with a gas that absorbs UV but not the infrared (IR) laser. \textcite{khanGasDetectionUsing2019} shows that \ce{NO2} and \ce{SO2} are two candidates. \ce{NO2} was first used as it was easy to produce in-house in significant quantities. The V1 system was set up with a vacuum pump connected to an outside air exhaust to safely vent the \ce{NO2} gas. The pump was also used to bring the pressure in the test section down to vacuum (how much?) before introducing gas.

            [Add absorption spectrum of NO2 from Gas Detection Using Portable Deep-UV Absorption Spectrophotometry: A Review or other place]

            Control LSP shots were undertaken in pure argon. Next, 0.55 bar of \ce{NO2}, or 200 mL at STP, were introduced into the chamber. It was then pressurized with argon to 20 bar. With the spark active, LSPs were consistently generated in the seeded atmosphere and their pressure trace from the PCB transducer was recorded with the oscilloscope. This pressure rise was approximately double the one seen in pure argon; see \autoref{fig:NO2_shots_analysis}.

            The next series of LSP shots was conducted with 0.24 bar (\qty{85}{ml} at STP) of \ce{NO2} and filled to 20.2 bar of argon. Again, higher pressure rises were observed, but slightly less than the \qty{0.55}{bar} shots. The chamber was finally half evacuated to 10.17 bar and then filled back to \qty{20.15}{bar}. This should bring the partial pressure of \ce{NO2} to \qty{0.12}{bar}. Again, LSPs were consistent, with a higher pressure rise than pure argon, but less than the higher concentration \ce{NO2} shots.

            \begin{figure}[h]
                \centering
                \includegraphics[width=0.75\textwidth]{assets/5 results/NO2_shots_analysis.pdf}
                \caption{NO2 LSP shots}
                \label{fig:NO2_shots_analysis}
            \end{figure}

            Indeed, with as low as (0.5\%?) of \ce{NO2} mixed with argon, nearly double the pressure rise is observed. This indicates that the working gas is absorbing twice the energy from the plasma. As the \ce{NO2} fraction is increased, there are diminishing returns to the pressure rise. This is encouraging, as not much \ce{NO2} is needed to have a great impact on the energy absorption.

    \section{V2 experiments and improvements}
        This section presents the various experiments that were conducted with the V2 thruster, as well as improvements that were made to this thruster along the way.
    
        \subsection{Initial LSP shots}

            \begin{figure}
                \centering
                \includegraphics[width=0.75\textwidth]{assets/5 results/V2 test damage.jpg}
                \caption{Damage to the thruster after two \qty{3}{kW} laser shots}
            \end{figure}

            % Stills of first High-speed LSP video, showing expansion of LSP wave like in V1
        
        \subsection{Cold flow thrust tests}

            Cold flow tests were completed to give a baseline measurement of thrust before the hot fire test, and to validate the functioning of all data acquisition systems.
            
            % thrust vs pressure graph

            To correct these problems, a more sensitive load cell was installed with a \qtyrange{0}{5}{N} force sensing range (Honeywell FSG005WNPB). However, the same issue 

            A different type of thurst stand (e.g. a rotating arm), could eventually be built to measure thrust in a more repeatable manner.

        \subsection{Bringing the pulsed power down, again}
            
            To prevent the damage to the thruster seen previously, a rear window mount was manufactured. This allows the laser energy that is not absorbed to pass freely through the apparatus, also enabling power meter measurements. 
            
            % Photo of rear window mount and damaged window

            As can be seen above, the window suffered laser damage after this round of testing. However, this damage proved minor as the window was used to align the laser focus with the spark gap, not to measure power absorption.
            
            The following figure present the LSP ignition attempts with focus distance, similarly to the graphs presented in \autoref{sec:pulse_power_down_V1}.

            \begin{figure}
                \centering
                \includegraphics[width=0.75\textwidth]{assets/5 results/V2_focus_threshold.pdf}
                \caption{LSP threshold graph for V2}
            \end{figure}

            The real amount of power in the pulsed shots was also measured to validate the \qty{11}{\%} threshold quoted previously. 10 shots each at \qtylist{10; 12}{\%} were measured with the power meter, with statistics compiled by the power meter software (see \autoref{label})

            \begin{figure}[h]
                \centering
                \begin{subfigure}[t]{0.45\textwidth}
                    \centering
                    \includegraphics[width=\textwidth]{assets/5 results/10p 50ms Statistics.png}
                    \caption{Statistics of 10 \qty{50}{ms} shots at 10\% power}
                \end{subfigure}
                \hfill
                \begin{subfigure}[t]{0.45\textwidth}
                    \centering
                    \includegraphics[width=\textwidth]{assets/5 results/12p 50ms Statistics.png}
                    \caption{Statistics of 10 \qty{50}{ms} shots at 12\% power}
                \end{subfigure}
                \caption{Test}
            \end{figure}
            
            At 10\% power, an 9.985 J average during \qty{50}{ms} gives an average power of \qty{200}{W}, lower than the expected \qty{300}{W} For \qty{12}{\%}. This revealed a higher power threshold, in terms of percentage, than previously thought. Extrapolating from these measurements, \qty{300}{W} is achieved at \qty{13.5}{\%}. This validated that the 

        \subsection{First CW LSP}
            
            A \qty{100}{\%} power CW shot was then attempted at \qty{20}{bar} argon. 

            % photo of laser on, Spark/LSP ignition, LSP constant, laser off (LSP dying down)

            This was the first CW LSP generated in the lab. No damage was noticed to the apparatus after this test.

        \subsection{Window extention tube}
            
            After the first CW LSP, more CW and pulsed shots were attempted. As laser alignment was not perfect, these continued to damage the rear window. Eventually, a \qty{3}{s} CW shot melted it severely enough that 

            \begin{figure}
                \centering
                \includegraphics[width=0.75\textwidth]{assets/5 results/window damage.jpg}
                \caption{Rear window damage on V2}
            \end{figure}

            The solution chosen was to manufacture a window extention tube, moving the rear window downstream to where the laser flux density is comparable to the front window, where no damage was seen.

            [drawing of extention tube]

        \subsection{New nozzle design}
            
            To solve the ablation of the aluminium nozzle under pulsed laser shots, the V2 thruster inner cylinder was modified and a new backplate was manufactured to accept graphite nozzle inserts. These cheap, chageable inserts are made from superfine iso-molded graphite rods sourced from Graphitestore (link with zoetro: https://www.graphitestore.com/isomolded-graphite-rod-0-500dia-x-12l-gt001685)

            [image of drawing highlighting changes]

        \subsection{CW power test}

            Wanted to see where exactly the pulsed power threshold was. For this, needed a max CW power measurement to compare against. 

            Tried to measure max CW power through two lenses, without the V2 apparatus. The two lenses were rated for this flux, but the 500 mm focal length lens shattered after about \qty{50}{s} of CW lasing. The leading hypothesis is that as the lens' temperature increased, it expanded, shattering it

            [graph of laser power that broke lens]

            \begin{figure}
                \centering
                \includegraphics[width=0.75\textwidth]{assets/5 results/Shattered 500 mm lens.jpg}
                \caption{Shattered 500 mm lens}
            \end{figure}

            This test showed a \qty{316}{W} (?) max CW power.

            Therefore, future CW tests will have to stay under this maximum time period, unless a lens cooling system is implemented. This could be as simple as a fan blowing cool air onto the lens.


        \subsection{CW LSP (hot fire) thrust tests}
 

            

        



