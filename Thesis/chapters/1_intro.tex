\chapter{Introduction} \label{chp:intro}
    
    \section{Motivation}

        In 2016, the Breakthrough Starshot initiative was proposed based on the work of \textcite{lubinRoadmapInterstellarFlight2016}. This mission involves sending \qty{1}{g} space probes to Alpha Centauri at \qty{20}{\%} the speed of light, using massive ground-based laser arrays. This could be enabled by a Moore's law in fiber laser technology, with a rapid doubling of power and a similar exponential decrease in costs. 
        
        As a near-term stepping stone using a smaller array, the laser can be coupled to a gas, reducing efficiency but increasing thrust. This would allow rapid interplanetary transfers (to Mars, notably) and is called Laser-Thermal Propulsion (LTP). The concept of LTP was first suggested by \textcite{kantrowitzRelevanceSpace1971} as a way to decrease launch costs and continues to be of interest. A conceptual design of an LTP spacecraft was proposed by \textcite{duplayDesignRapidTransit2022a}, with a similar architecture to Breakthrough Starshot: a \qty{10}{m} laser array beams \qty{100}{MW} of power to an orbiting spacecraft for injection burns. With a \qty{1}{ton} payload, \qty{6}{kN} of thrust and \qty{3000}{s} of $I_\mathrm{sp}$, a \qty{1}{h} laser beaming maneuver gives \qty{14}{km/s} of delta-V to the spacecraft, which reaches Mars in 45 days.

        \begin{figure}[!ht]
            \centering
            \includegraphics[width=0.4\textwidth]{assets/2 background/ltp_architecture.pdf}
            \caption{LTP architecture (\textcite{duplayArgonLaserPlasmaThruster2024a})}
            \label{fig:LTP architecture}
        \end{figure}

        In the thrust chamber of the vehicle, hydrogen is introduced. The laser is focused inside the chamber and is absorbed by the gas via inverse Bremsstrahlung, creating a Laser-Supported Plasma (LSP) core. Colder hydrogen flows around the LSP core and is heated by it to \qty{10000}{K}. The hot gas is then exhausted though a conventional converging-diverging nozzle at the exhaust velocity, imparting thrust to the vehicle.

        \begin{figure}[!ht]
            \centering
            \includegraphics[width=0.6\textwidth]{assets/2 background/chamber.pdf}
            \caption{Overview of LTP system (\textcite{duplayArgonLaserPlasmaThruster2024a})}
            \label{fig:LTP system overview}
        \end{figure}

        In a conventional chemical rocket engine, the energy source is the oxidizer and the fuel, which are reacted together to release energy. They are transported with the rocket and set the temperature of the combustion reaction (typically \qtyrange{2000}{3000}{K}), which is directly related to the exhaust velocity.
        
        Separating the power source used for propulsion (here, the laser) from the spacecraft itself allows crucial weight savings, either increasing the payload mass fraction or decreasing transit time. Using a laser also allows for much greater thrust chamber temperatures than chemical propulsion, as the temperature of these plasmas is typically \qtyrange{15000}{30000}{K}. This gives in turn greater exhaust velocities. This propulsion method could therefore be an order of magnitude more efficient than our current rocket engines if certain engineering problems can be solved.

        Increasing the amount of energy deposited by the laser into the working gas remains a topic of active research and is a significant hurdle for the operational use of LTP. The two main conversion efficiencies are:
        \begin{enumerate}
            \item Absorption of the laser energy by the plasma
            \item Heat transfer from the plasma to the working gas (e.g. propellant)
        \end{enumerate}
        A selection of past LSP experiments will now be presented, with an emphasis on these efficiencies. As the efficiencies chosen are different from source to source, these will be defined where applicable.
    
    \section{Literature review}

        %Russian work: 1960s-2000s?

        The experimental basis of LTP was developed by \textcite{generalovContinuousOpticalDischarge1970} in 1970. For the first time, an LSP was generated with a \qty{150}{W} \ce{CO2} laser operating at \qty{10.6}{μm} wavelength. In this case, the LSP was initiated by a second, \qty{10}{kW} pulsed \ce{CO2} laser.
        
        %American work: 1970s-1990s?

        Work was done in the mid-1970s by \textcite{shojiLaserheatedRocketThruster1977,shojiPerformanceHeatTransfer1976a} to design a small-scale \qty{10}{kW} and full-scale \qty{5000}{kW} LTP engine. Carbon-seeded hydrogen was chosen to capture the plasma's radiation, which was mostly in the UV wavelength. \qty{19}{\%} of the laser power would be lost by radiation in the \qty{10}{kW} thruster, while the \qty{5000}{kW} thruster would lose \qty{4.5}{\%} of laser power by radiation. However, this was not tested. The \qty{10}{kW} prototype (\autoref{fig:Shoji apparatus}) was built and delivered to NASA at the conclusion of their effort.
        \begin{figure}[!ht]
            \centering
            \begin{subfigure}[t]{0.45\textwidth}
                \centering
                \includegraphics[width=\textwidth]{assets/2 background/Shoji_assy.png}
                \caption{\qty{10}{kW} thruster from \textcite{shojiPerformanceHeatTransfer1976a}}
                \label{fig:Shoji apparatus}
                \end{subfigure}
            \hfill
            \begin{subfigure}[t]{0.45\textwidth}
                \centering
                \includegraphics[width=\textwidth]{assets/2 background/Shoji cross-section.png}
                \caption{Cross-section drawing of \qty{10}{kW} thruster from \textcite{shojiLaserheatedRocketThruster1977} (original is of poor quality)}
                \label{fig:Shoji cross-section}
            \end{subfigure}
            \caption{Thruster designed by \textcite{shojiLaserheatedRocketThruster1977}}
            \label{fig:Shoji apparatussies}
        \end{figure}

        In the 1980s, \textcite{keeferPowerAbsorptionLasersustained1986a} studied LSP in a forced convective flow environment. Using a \qty{1.5}{kW} \ce{CO2} laser with power levels of \qtyrange{360}{840}{W} and pressures of \qtyrange{1.3}{2.3}{atm}, with varying argon flow velocities, the temperature field of the plasma was measured. From the temperature field, and assuming local thermodynamic equilibrium, the power absorbed by the plasma and the power radiated from it can be calculated.
        \begin{figure}[!ht]
            \centering
            \includegraphics[width=0.4\textwidth]{assets/2 background/UTSI (Keefer) Apparatus.png}
            \caption{Experimental apparatus from \textcite{keeferPowerAbsorptionLasersustained1986a}}
            \label{fig:Keefer apparatus}
        \end{figure}
        \autoref{fig:Keefer apparatus} shows the apparatus used for these measurements. An inner quartz flow channel contained the plasma, while an outer quartz jacket contained the pressure. The plasma was initiated by laser heating of a tungsten rod, which was removed after initiation. Downstream, a water-cooled copper beam dump absorbed the laser energy and cooled the heated argon flow. The power absorbed by the plasma was between \qtyrange{23}{61}{\%} of incident laser power, while radiation loss was between \qtyrange{51}{80}{\%} of the absorbed power.

        Contemporary to \textcite{keeferPowerAbsorptionLasersustained1986a}, Mazumder and Krier headed a group at the University of Illinois that advanced the field of LTP. \textcite{krierContinuousWaveLaser1986a} reported laser absorption in an argon plasma approaching \qty{80}{\%}. \autoref{fig:Krier apparatus} shows the apparatus that was used by \textcite{krierContinuousWaveLaser1986a}, \textcite{zerkleLasersustainedArgonPlasmas1990} and \textcite{chenEmissionSpectroscopyCw1989a}. This vertical cylindrical flow chamber was made of 304 steel and had an internal diameter of 5 inches.  A water-cooled calorimeter was used as a beam dump for the \qty{10}{kW} \ce{CO_2} laser. The laser energy not collected by the beam dump was assumed to be absorbed by the plasma, as reflection losses are less than \qty{2}{\%} at these electron number densities. Moveable thermocouples gave two-dimensional maps of the flow surrounding the plasma core. The minimum maintenance intensity of the plasma was also estimated at \qtyrange{0.1}{0.3}{MW/cm^2}.
        \begin{figure}[!ht]
            \centering
            \includegraphics[width=0.5\textwidth]{assets/2 background/Illinois (Krier) Apparatus.png}
            \caption{Experimental apparatus from \textcite{zerkleLasersustainedArgonPlasmas1990}}
            \label{fig:Krier apparatus}
        \end{figure}
        Thermal efficiency was between \qtyrange{6}{25}{\%}, with radiative losses of \qty{64}{\%} and \qty{30}{\%}, respectively. Thermal efficiency was defined as:
        \[\eta_\mathrm{th} =  \frac{\text{Power retained by the gas}}{\text{Incident laser power}}\]
        Further work by \textcite{zerkleLasersustainedArgonPlasmas1990} with this apparatus (\autoref{fig:Krier apparatus}) reported absorption from \qtyrange{55}{97}{\%} and thermal efficiency from \qtyrange{11}{46}{\%}. This was done in \qtylist{1 2.5}{atm} of flowing argon, with laser powers up to \qty{7}{kW}. \textcite{chenEmissionSpectroscopyCw1989a} again increased the thermal efficiency of this apparatus, with \qtyrange{41}{62}{\%} of the laser energy being retained by the gas as thermal energy. This was among the highest thermal efficiencies measured by an LSP experiment. Here, \qty{86}{\%} of the laser's energy is absorbed by the LSP. This was attained with a \qty{5}{kW} \ce{CO2} laser, with flow speeds between \qtyrange{2}{10}{m/s}. They discuss that greater thermal efficiency is due to greater laser power, a high enough flow speed and a greater laser focusing $f$ number.

        Based on work by the Illinois group, an LTP engine demonstrator was tested in 1995 by \textcite{blackLaserPropulsion10kW1995} with a \qty{10}{kW} \ce{CO2} laser. This was planned to be a step towards a full-scale thruster. More than 100 thruster firings were completed, lasting 1 to 2 minutes each. The \qty{10}{kW} thruster is presented in \autoref{fig:Black apparatus}. It is mounted in a vacuum chamber to a thrust measurement assembly. Efficiency was calculated from:
        \[ \eta = \frac{F^2}{2 \dot{m} P_\mathrm{L}} \]
        with $P_\mathrm{L}$ the input laser power at the thruster window. Both argon and hydrogen were used. Argon propellant produced \qty{200}{s} of $I_\mathrm{sp}$ and a peak efficiency of 0.24. With hydrogen propellant, an $I_\mathrm{sp}$ of \qty{350}{s} and a peak efficiency of 0.37 were reported.
        \begin{figure}[!ht]
            \centering
            \begin{subfigure}[t]{0.3\textwidth}
                \centering
                \includegraphics[width=\textwidth]{assets/2 background/BlackKrier thruster.png}
                \caption{\qty{10}{kW} thruster}
                \label{fig:Black apparatus}
            \end{subfigure}
            \hfill
            \begin{subfigure}[t]{0.45\textwidth}
                \centering
                \includegraphics[width=\textwidth]{assets/2 background/Black thrust measurement assy.png}
                \caption{Thrust measurement assembly}
                \label{fig:Black thrust measurement}
            \end{subfigure}
            \caption{Apparatus used in \textcite{blackLaserPropulsion10kW1995}}
            \label{fig:Black apparatussies}
        \end{figure}
        A preliminary design for the full-scale \qty{100}{kW} thruster was also presented, with a predicted specific impulse of \qty{1000}{s}, thrust of \qty{4.5}{N} and a conversion efficiency of \qty{80}{\%}.

        %Chinese and Japanese work: 1980s-2020s

        In the early 2000s, \textcite{toyodaThrustPerformanceCW2002} built and tested two different thruster models, presented in \autoref{fig:Toyoda apparatussies}. These thrusters, using argon or nitrogen heated by LSP, were powered by a \qty{2}{kW} \ce{CO2} laser. The LSPs were initiated by a retractable tungsten rod at the laser's focus. Thrust measurements were done both in atmospheric pressure and in vacuum. This comparative study showed that confining the plasma into a smaller chamber increased heat transfer and therefore, efficiency. \textcite{toyodaThrustPerformanceCW2002} defined the energy conversion efficiency as the amount of laser power that is converted into usable kinetic energy for thrust. It is calculated as\footnote{As mentioned by \textcite{duplayArgonLaserPlasmaThruster2024a}, there appears to be a typographical error in the reference as the units are inconsistent. The corrected equation is presented here.}:
        \[ \eta_\mathrm{e} =  \frac{F^2_\mathrm{hot} -F^2_\mathrm{cold}}{2 \dot{m} P}\]
        Where $F_\mathrm{hot}$ is the thrust with laser on, $F_\mathrm{cold}$ is the cold flow thrust (laser off) and $P$ is incident laser power. An energy conversion efficiency of 37\% and an $I_\mathrm{sp}$ of \qty{113}{s} were measured with the second model \autoref{fig:Toyoda apparatus 2} in vacuum with argon propellant. The pressure ratio, defined as the chamber pressure divided by the nozzle exit pressure, was 420. A water cooling system measured the heat loss to the walls to be 55\% of incident laser power, with a final 8\% being ``other loss". Heat loss to the walls was expected to be recycled with regenerative cooling in a real-world application.

        \begin{figure}[!ht]
            \centering
            \begin{subfigure}[t]{0.45\textwidth}
                \centering
                \includegraphics[width=\textwidth]{assets/2 background/Toyoda apparatus model 1.jpg}
                \caption{Model I}
                \label{fig:Toyoda apparatus 1}
            \end{subfigure}
            \hfill
            \begin{subfigure}[t]{0.45\textwidth}
                \centering
                \includegraphics[width=\textwidth]{assets/2 background/Toyoda Apparatus model 2.png}
                \caption{Model II}
                \label{fig:Toyoda apparatus 2}
            \end{subfigure}
            \caption{Two thruster models from \textcite{toyodaThrustPerformanceCW2002}}
            \label{fig:Toyoda apparatussies}
        \end{figure}

        \textcite{luCharacteristicDiagnosticsLaserStabilized2022a} investigated LSP for lighting applications instead of propulsion. Therefore, an emphasis was made on spectroscopy measurements. A \qty{300}{W} fiber laser at a wavelength of \qty{1080}{nm} was focused to a \qty{50}{μm} diameter spot in a high pressure chamber.
        \begin{figure}[!ht]
            \centering
            \includegraphics[width=0.7\textwidth]{assets/2 background/Lu apparatus.png}
            \caption{Experimental setup from \textcite{luCharacteristicDiagnosticsLaserStabilized2022a}}
            \label{fig:Lu apparatus}
        \end{figure}
        Argon was used, with pressures between \qtyrange{10}{20}{bar}. A lower initiation power (\qty{117}{W}) than other studies was achieved at \qty{20}{bar}. This was attributed to the smaller focus delivering a greater photon flux. \ce{N2} was later added between \qtyrange{0.1}{1.0}{\%}. As expected, increasing the laser power or the gas pressure was found to increase the radiation intensity of the LSP. However, adding \ce{N2} reduced both the electron temperature and electron density of the LSP, reducing its radiation intensity.

        Seeding the working gas with another species has been discussed as a way to increase the energy absorption into the working fluid of an LTP engine. LSPs in pure methane and methane-seeded gasses have been investigated by \textcite{kameiMethaneMethaneXenon2020}. Methane dissociates into hydrogen and carbon with the high temperature of the LSP. As mentioned with \textcite{shojiLaserheatedRocketThruster1977}, carbon particles would absorb the LSP's UV radiation.  A \qty{1.1}{kW} diode laser at a wavelength of \qty{940}{nm} was beamed into a high-pressure chamber fitted with arc initiation electrodes. The gap between these electrodes was \qty{1}{mm}. A CCD type spectrometer recorded emission spectra of the initiation arc discharge and of the LSP. LSPs in three different gasses were attempted: pure methane, methane-argon, and methane-xenon.
        \begin{figure}[!ht]
            \centering
            \includegraphics[width=0.7\textwidth]{assets/2 background/Kamei apparatus.png}
            \caption{LSP generation chamber and cross-section view from \textcite{kameiMethaneMethaneXenon2020}}
            \label{fig:Kamei}
        \end{figure}
        In methane at \qty{0.1}{MPa}, soot formation between the electrodes prevented LSP initiation. The spectrometer confirmed the dissociation of methane, as line spectra of carbon and hydrogen were observed at the initiation arc. Initiation was also unsuccessful in argon-methane with a pressure between \qtyrange{0.1}{0.3}{MPa} and a methane volume fraction between \qtyrange{20}{60}{\%}. LSP was successfully generated in methane-xenon, with a lower threshold power (\qty{850}{W}) than in pure xenon. The partial pressure of methane was between \qtyrange{0.02}{0.6}{MPa}, with a partial pressure of xenon of \qty{0.10}{MPa}.

        \textcite{takanoDemonstrationDiodeLasersustained} used a diode laser emitting simultaneously at \qty{927}{nm} and \qty{951}{nm} to generate LSPs in argon. This resulted in an $I_\mathrm{sp}$ of \qty{105}{s} and a thrust efficiency of \qty{8}{\%}. They define thrust efficiency as:
        \[
        \eta = \frac{g_0 I_\mathrm{sp} (F_\mathrm{hot}-F_\mathrm{cold})}{2 P_\mathrm{laser}}
        \]
        Two setups were used: the LSP generation chamber previously used by \textcite{kameiMethaneMethaneXenon2020} (\autoref{fig:Takano LSP generation chamber}) and an LSP thruster (\autoref{fig:Takano LSP thruster}).
        \begin{figure}[!ht]
            \centering
            \begin{subfigure}[t]{0.45\textwidth}
                \centering
                \includegraphics[width=\textwidth]{assets/2 background/Takano LSP chamber.png}
                \caption{LSP generation chamber and systems}
                \label{fig:Takano LSP generation chamber}
            \end{subfigure}
            \hfill
            \begin{subfigure}[t]{0.45\textwidth}
                \centering
                \includegraphics[width=\textwidth]{assets/2 background/Takano LSP thruster.png}
                \caption{LSP thruster}
                \label{fig:Takano LSP thruster}
            \end{subfigure}
            \caption{LSP setups from \textcite{takanoDemonstrationDiodeLasersustained}}
            \label{fig:Takano apparatussies}
        \end{figure}
        The LSP chamber was used to determine the effect of various F-numbers on the argon LSP. The thruster has an interchangeable copper throat, with diameters of \qty{0.7}{mm} and \qty{1.0}{mm}. In both setups, electric arc initiation was used. Once initiated in the thruster, the LSP is moved toward the nozzle with the lens mounted on a motorized stage. It was found that moving the LSP this way increased the heat exchange with the working gas. Thrust was calculated by using the pressure measurements inside the thruster's heating chamber.

        %Canadian work: 2020s  

        \textcite{duplayArgonLaserPlasmaThruster2024a} used a \qty{3}{kW} pulsed fiber laser to create LSPs in static and flowing argon. In static argon, about \qty{80}{\%} of the laser energy was being absorbed by the plasma, with approximately \qty{15}{\%} of the laser energy heating the bulk gas. This was done between \qtyrange{5}{20}{bar}.
        \begin{figure}[!ht]
            \centering
            \includegraphics[width=0.7\textwidth]{assets/2 background/finalsetup_static.pdf}
            \caption{Static LSP apparatus from \textcite{duplayArgonLaserPlasmaThruster2024a}}
            \label{fig:Duplay apparatus}
        \end{figure}

    \section{Summary and direction of work in this thesis}
        
        From the literature review, \autoref{tab:lit review summary} and \autoref{tab:efficiencies} were compiled. Most studies have used \ce{CO2} lasers with a wavelength of \qty{10.6}{μm}. However, these concepts were limited by a short focusing range due to their long laser wavelength. 
        
        Indeed, the diffraction limit of a laser, which is the theoretical lower limit on beam divergence, equals the wavelength ($\lambda$) divided by the diameter $D$ of the output beam \cite{hechtUnderstandingLasersEntry2019}.
        \[
        \text{Diffraction limit (radians)} = \lambda/D
        \]
        This relegated \ce{CO2} lasers to ground-to-orbit launch. Recently, high power fiber lasers emitting near \qty{1}{μm} have become readily available. Being able to beam energy to low earth orbit, fiber lasers make laser propulsion more feasible.

        \begin{table}[!ht]
            \centering
            \caption{Summary of a selection of past LSP experiments. $\lambda$: wavelength, $P$: maximum laser power, $p$: pressure, $\dot m$: mass flow rate, $I_\mathrm{sp}$: maximum specific impulse, $F_\mathrm{T}$: maximum thrust}
            \label{tab:pastexp}
            \begin{tabularx}{\textwidth}{@{}>{\small}X<{\raggedright}lXXrXXXrr<{\raggedright}@{}}
            \toprule
            {\normalsize LSP   Facility} & Year & Laser & $\lambda$ [\unit{\um}] & $P$ [kW] & Gas & $p$ [atm] & $\dot m$ [g/s] & $I_\mathrm{sp}$ [s] & $F_\mathrm{T}$ [N]  \\ \midrule
            \textcite{generalovContinuousOpticalDischarge1970}        &1970&\ce{CO_2}&10.60&0.15 &\ce{Xe}           & 3.0-4.0  &  -  & -   &  -   \\
            \textcite{keeferPowerAbsorptionLasersustained1986a}       &1986&\ce{CO_2}&10.60&0.84 &\ce{Ar}           & 1.3-2.3  & 0.01-0.19   & -   &  -   \\
            \textcite{krierContinuousWaveLaser1986a}                  &1986&\ce{CO_2}&10.60&10  &\ce{Ar}           &     -       & 2.3-4.6 & -& -\\
            \textcite{zerkleLasersustainedArgonPlasmas1990}           &1988&\ce{CO_2}&10.60&7   &\ce{Ar}           &      -      &     -     & -&- \\
            \textcite{chenEmissionSpectroscopyCw1989a}                &1989&\ce{CO_2}&10.60&5   &\ce{Ar}           &      -      &  -  & -   & -    \\
            \textcite{blackLaserPropulsion10kW1995}                   &1995&\ce{CO_2}&10.60&10  &\ce{Ar}           & 1.4-2.4  & 5.1-9.4 & 200 & 7 \\
                                                                      &    &\ce{CO_2}&10.60&10  &\ce{H_2}          &    3.4        & 1.1&  350   & 3 \\
            \textcite{toyodaThrustPerformanceCW2002}                  &2002&\ce{CO_2}&10.60&2   &\ce{Ar}, \ce{N_2} & 2.0-5.5  &  -  & 113 & 0.44 \\
            \textcite{luCharacteristicDiagnosticsLaserStabilized2022a}&2022&Fiber    &1.08 &0.30 &\ce{Ar}, \ce{N_2} & 9.9-19.7 &  -  & -   & -    \\ 
            \textcite{takanoDemonstrationDiodeLasersustained}         &2024&Diode    &0.927 and 0.951 &4.4 &\ce{Ar}   & 10-15 &   - & 105   & -    \\
            \textcite{duplayArgonLaserPlasmaThruster2024a}            &2024&Fiber    &1.07  & 3 &\ce{Ar}            &   5-20  & 0 & - & - \\
            \bottomrule
            \end{tabularx}
            \label{tab:lit review summary}
        \end{table}

        \begin{table}[!ht]
            \centering
            \caption{Comparative table of experimental LTP thruster efficiencies}
            \label{tab:efficiencies}
            \begin{tabularx}{\textwidth}{@{}>{\small}X<{\raggedright} r l r@{}}
            \toprule
            {\normalsize LSP   Facility}   & Laser absorption  & Efficiency & Value of efficiency \\ \midrule
            \textcite{keeferPowerAbsorptionLasersustained1986a}   & 0.23 - 0.61      &          -        &                 -          \\
            \textcite{krierContinuousWaveLaser1986a}       & 0.50 - 0.80         & $\eta_\mathrm{th} =  \frac{\text{Power retained by the gas}}{\text{Incident laser power}}$ &  0.06 - 0.25 \\
            \textcite{zerkleLasersustainedArgonPlasmas1990}       & 0.55 - 0.97   &         $\eta_\mathrm{th} =  \frac{\text{Power retained by the gas}}{\text{Incident laser power}}$ &  0.11 - 0.46 \\
            \textcite{chenEmissionSpectroscopyCw1989a}          & 0.86                      &  $\eta_\mathrm{th} =  \frac{\text{Power retained by the gas}}{\text{Incident laser power}}$  &  0.41-0.62  \\
            \textcite{blackLaserPropulsion10kW1995}       &  -  & $ \eta = \frac{F_\mathrm{hot}^2}{2 \dot{m} P_\mathrm{L}} $& 0.20 - 0.25 (Ar), 0.25 - 0.40 (H)     \\
            \textcite{toyodaThrustPerformanceCW2002}    & -                      & $ \eta_\mathrm{e} =  \frac{F^2_\mathrm{hot} -F^2_\mathrm{cold}}{2 \dot{m} P} $     &   0.37 \\
            \textcite{takanoDemonstrationDiodeLasersustained}  &       -       & $ \eta = \frac{g_0 I_\mathrm{sp} (F_\mathrm{hot}-F_\mathrm{cold})}{2 P_\mathrm{laser}} $ & 0.08 \\ 
            \textcite{duplayArgonLaserPlasmaThruster2024a}  &  0.80&  $\eta_\mathrm{th} =  \frac{\text{Power retained by the gas}}{\text{Incident laser power}}$ & 0.15 \\
            \bottomrule
            \end{tabularx}
        \end{table}
        

        To increase thermal efficiency, \textcite{chenEmissionSpectroscopyCw1989a} suggest:
        \begin{enumerate}
            \item A greater laser power, which gives greater inverse bremsstrahlung absorption coefficient and longer absorption path length;
            \item A high enough flow speed to push the LSP back to the laser focus, but not too fast as to blow the plasma out;
            \item A greater laser focusing $f$ number, creating a longer and narrower plasma. This increases the probability a photon will be absorbed by the plasma and reduces the radiation loss.
        \end{enumerate}

        For a small-scale demonstration thruster, $I_\mathrm{sp}$ values near \qty{100}{s} can be expected, with thrust values under \qty{1}{N}, as was found by \textcite{toyodaThrustPerformanceCW2002} and \textcite{takanoDemonstrationDiodeLasersustained}.

        The objective of this research project will be to test a lab-scale proof-of-concept LTP thruster for interplanetary space flight, using a \qty{1.07}{μm} fiber laser. Key performance parameters of this thruster such as thrust and specific impulse will be measured, and the LSP heating core inside the thruster will be characterized. This project will mainly build upon the experimental research started by \textcite{duplayArgonLaserPlasmaThruster2024a}.
        
        % Link to next chapter, modelling c_p
        Before going into experiments, it is important to characterize certain thermodynamic properties of the gasses used. Notably, the modelling of heat capacity will be presented in the next chapter.
