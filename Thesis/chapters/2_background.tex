\chapter{Background} \label{chp:background}
    \section{Russian work: 1960s-2000s?}
    \section{American work: 1970s-1990s?}
    \section{Japanese work: 1980s-2020s?}
    \section{Chinese work: 2020s?}
    \section{Canadian work: 2020s}
    \section{Summary and direction of work in this thesis}

        \begin{table}[h]
            \centering
            \caption{Summary of a selection of past LSP experiments. $\lambda$: wavelength, $P$: maximum laser power, $p$: pressure, $I_\mathrm{sp}$: maximum specific impulse, $F_\mathrm{T}$: maximum thrust}
            \label{tab:pastexp}
            \begin{tabularx}{\textwidth}{@{}>{\small}X<{\raggedright}llrrlrrr>{\footnotesize}X<{\raggedright}@{}}
            \toprule
            {\normalsize LSP   Facility}                                                           & Year & Laser         & $\lambda$   [\unit{\um}] & $P$ [kW] & Gas                 & $p$   [atm] & $I_\mathrm{sp}$ [s] & $F_\mathrm{T}$   [N] & {\normalsize Comments}                                                                   \\ \midrule
            \textcite{generalovContinuousOpticalDischarge1970}                                                         & 1970 & \ce{CO_2}                  & 10.60             & 0.15               & \ce{Xe}              & 3.0 - 4.0        &           -             &       -       & First   LSP                                                                \\
            \textcite{keeferPowerAbsorptionLasersustained1986}         & 1986 & \ce{CO_2}                  & 10.60             & 0.84        & \ce{Ar}              & 1.3   - 2.3      &            -            &       -       & Specialized   laser beam dump integrated within the converging exit nozzle \\
            \textcite{blackLaserPropulsion10kW1995}          & 1995 & \ce{CO_2}                  & 10.60             & 10.00              & \ce{Ar},   \ce{H_2}    & 1.0   - 2.7      & 350                    & 3.00         & 15:1   expansion ratio nozzle                                              \\
            \\
            \textcite{toyodaThrustPerformanceCW2002} & 2002 & \ce{CO_2}                  & 10.60             & 2.00               & \ce{Ar},   \ce{N_2}    & 2.0   - 5.5      & 113             & 0.44         & Tungsten   rod ignition                                                    \\
            \textcite{zimakovInteractionNearIRLaser2016}                                                        & 2016 & Fiber                & 1.07              & 1.50        & \ce{Ar},   \ce{Xe}   & 3.0   - 24.7     & -                      & -            & Arc   discharge ignition                                                   \\
            \textcite{matsuiGeneratingConditionsArgon2019}                                                          & 2019 & Fiber &      1.07             &        2.00            &           \ce{Ar}           &          1.0 - 64.2        &            -            &       -       & Arc   discharge ignition                                                   \\
            \textcite{luCharacteristicDiagnosticsLaserStabilized2022}                                                             & 2022 & Fiber                & 1.08              & 0.30      & \ce{Ar},   \ce{Ar + N_2} & 9.9   - 19.7     & -                      & -            & Arc   discharge ignition                                                   \\ \bottomrule
            \end{tabularx}
        \end{table}

        The original contribution of this work 


        % Higgins text from email to Mark Wolverton: Laser thermal propulsion originated with Arthur Kantrowitz, who was involved in developing the first gas dynamic lasers that would be able to reach MW-class power in the 1960s. (Interesting connection to Jordin Kare: Jordin was also a filk-singer, which is a science-fiction themed version of folk singing. He actually wrote a song about laser thermal propulsion called “Kantrowitz 1972”. Put your coffee cup down before listening to this…) 
        %The laser-thermal rocket kind of died with the demise of high-power laser weapon research in the early 1990s. Now with Phil Lubin’s work, I think it is worthwhile looking at again. Even with a gigawatt-class laser, laser propulsion is not well suited for ground-to-orbit launch vehicles, unless they are very small (this is what Jordin’s filk song is about—rapid turn around of launching many microlaunchers using a laser).  Here’s the clip of Elon Musk making the same point: https://youtu.be/viRylmoFAj0?si=MzzsBkhHF71FLCRG
        %However, with Philip Lubin’s demonstration that phased array lasers can be made arbitrarily large, we can now reach much deeper into space and perform propulsive maneuvers more leisurely (say, over hours), so the power requirements now drop to the 100s of MW (rather than 10s of gigawatts!). This was the basis of our 45-days-to-Mars study, which I think you have already seen. Here is an un-paywalled version of it: [2201.00244] Design of a rapid transit to Mars mission using laser-thermal propulsion (arxiv.org)
        %The other approach is to provide laser power onto a solar panel (although one tuned to the laser wavelength) that then powers electric propulsion like an ion engine or Hall effect thruster: Laser-electric propulsion. Lubin’s group published a similar study to ours, trying to hit similar objectives (the usual “Mars in a month” metrics). I’ve attached their paper here: SheerinLubinEtAl_FastTransporationElectricPropulsionDirectedEnergy_ActaAstro2021
        %Their paper and ours make for a nice “compare and contrast” of the pluses and minuses of the laser-electric and laser-thermal approaches.
